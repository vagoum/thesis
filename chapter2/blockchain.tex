
Η τεχνολογία του blockchain παρουσιάστηκε για πρώτη φορά από τον Satoshi
Nakamoto το 2008 \cite{bitcoin}, και αποτελεί την πρώτη υλοποίηση ψηφιακού
νομίσματος, που λύνει αποκεντρωμένα το πρόβλημα του διπλοξοδέματος (double
spending).  Αποτελεί ένα μέσο επίλυσης χρηματικών συναλλαγών χωρίς την επέμβαση
κάποιας κεντρικής αρχής, που βασίζεται σε κρυπτογραφικές μεθόδους για την
επίτευξη της συναίνεσης των συμμετεχόντων.

Σύντομα μετά την εν λόγω δημοσίευση που περιγράφει τις αρχές λειτουργίας του
bitcoin και εισάγει την έννοια του blockchain, δημοσιέυθηκε ως λογισμικό
ανοιχτού κώδικα η υλοποίηση του συστήματος σε C++ το 2009, πάλι από τον Satoshi
Nakamoto και την ομάδα του, ο οποίος λίγο αργότερα εξαφανίστηκε χωρίς να
γνωρίζουμε την πραγματική του ταυτότητα. μέχρι και σήμερα.


Η δημιουργία ενός ψηφιακού νομίσματος ελεύθερο από τον έλεγχο μιας κεντρικής
αρχής απασχολούσε αρκετά χρόνια την κοινότητα των cypherpunks την δεκαετία του
90', όπου μπήκαν τα θεμέλια για την εισαγωγή της τεχνολογίας του blockchain από
τον Nakamoto, όπως η χρήση των Merkle trees και των ψηφιακών υπογραφών.

Απο τότε, η τεχνολογία του blockchain έχει προσελκύσει μεγάλο ενδιαφέρον λόγω
των εφαρμογών της και θεωρείται από τις πιο πρωτοποριακές τεχνολογικές
εφευρέσεις του 21ου αιώνα.


\section{Έξυπνα συμβόλαια}

Η έννοια των έξυπων συμβολαίων (smart contracts), προηγείται αυτής του Bitcoin
και μπορεί να εντοπιστεί στο 1997, από τον Nick Szabo \cite{szabo1}, ο οποίος
οραματίστηκε αρκετές από τις σημερινές εφαρμογές των έξυπνων συμβολαίων, όπως η
εκκαθάριση πληρωμών, η διαχείριση και μεταφορά ιδιοκτησίας,  η εμπιστοσύνη σε
ανώνυμα και ψευδώνυμα δίκτυα, και η αποκεντρωμένη εκδοχή κλασσικών
χρηματοοικονομικών συμβολαίων, όπως τα παράγωγα. Προτείνει ακόμα μία γλώσσα για
χρήση στον προγραμματισμό έξυπνων συμβολαίων, βαθιά επηρεασμένη από τον
συναρτησιακό προγραμματισμό \cite{szabosmartcontract}.  \\ Ο όρος ``έξυπνο
συμβόλαιο'' απέκτησε νέα σημασία με την εισαγωγή της τεχνολογίας του blockchain.
Η πρώτη πλατφόρμα που υποστήριξε έξυπνα συμβόλαια ήταν το Ethereum.

\subsection{Ethereum}

Η πλατφόρμα του Ethereum έχει αρκετά κοινά με το Bitcoin, όπως ότι η λειτουργία
του βασίζεται σε ένα \textit{αποκεντρωμένο δίκτυο κόμβων}, και στην ύπαρξη ενός
\textit{αλγορίθμου συναίνεσης} (consensus) μεταξύ των κόμβων, που τους
επιτρέπει να διατηρήσουν μια κατανεμημένη βάση δεδομένων. Στην περίπτωση του
Bitcoin η δομή αυτή συμφωνεί στο ``ποιος έχει τι''.

Το Ethereum επεκτείνει την ιδέα αυτή, χτίζοντας πάνω στην τεχνολογία των
κρυπτονομισμάτων της εποχής.  Κάνοντας χρήση του στρώματος συναίνεσης που
μοιράζει σωστά και δίκαια τους πόρους του δικτύου, και υποστηρίζει έξυπνα
συμβόλαια με αρκετά μεγαλύτερη εκφραστικότητα. Σε αντίθεση με το Bitcoin, του
οποίου η γλώσσα επαλήθευσης των συναλλαγών περιορίζεται στο να επαληθεύει αν
ισχύουν οι συνθήκες που επιτρέπουν το ξόδεμα των πόρων, η γλώσσα του Ethereum
είναι Turing-complete. Αυτό πρακτικά σημαίνειότι τα έξυπνα συμβόλαια μπορούν
πλέον να εκφράσουν οποιονδήποτε υπολογισμό μπορεί να γίνει με μία mainstream
γλώσσα προγραμματισμού. Το γεγονός αυτό έχει δώσει στο Ethereum την ονομασία
``κατανεμημένος παγκόσμιος υπολογιστής''.

\subsection{Εφαρμογές}

Οι δυνατότητες που παρέχει η τεχνολογία του blockchain σε συνδυασμό με έξυπνα
συμβόλαια γενικού σκοπού βρίσκουν πληθώρα εφαρμογών σε διάφορες βιομηχανίες και
κλάδους. Το κύριο χαρακτηριστικό τους που κάνει δελεαστική την χρήση τους σε
παραδοσιακές βιομηχανίες είναι ότι αποτελούν συμβόλαια που μπορούν να
εκτελεστούν και να επιβληθούν αυτόματα χωρίς την επέμβαση μεσάζοντα,
απλοποιώντας έτσι πολλές πτυχές των σημερινών βιομηχανιών όπως στον
\textit{εφοδιασμό} (supply chain) σε \textit{συναλλαγές και πληρωμές},
\textit{μεταφορά ακινήτων}, \textit{ασφάλεια - υπηρεσίες υγείας} και
\textit{ηλεκτρονικές ψηφοφορίες}

Οι παραπάνω βιομηχανικοί κλάδοι αποτελούν παράδειγμα περιπτώσεων χρήσης που
μπορούν να ωφεληθούν μειώνοντας τα κόστη λειτουργίας μέσω της χρήσης έξυπνων
συμβολαίων. Υπάρχουν όμως και εφαρμογές που έχουν γίνει δυνατές μόνο μέσω της
νέας τεχνολογίας αυτής, όπως οι \emph{αγορές προβλέψεων} (prediction markets)
(\cite{augur}) και οι \emph{αποκεντρωμένοι αυτόνομοι οργανισμοί} (DAO).

Οι χρήσεις αυτές των έξυπνων συμβολαίων αφορούν παραδοσιακούς τομείς που
ωφελούνται σε μεγάλο βαθμό από την πιο αποδοτικότερη εκτέλεση των συμβολαίων -
συμφωνιών, αλλά και δίνουν δυνατότητες για δημιουργία εφαρμογών με γνώμονα την
ιδιωτικότητα, την ασφάλεια, την ακρίβεια, την ταχύτητα και την διαφάνεια.

Είναι αναγκαίο επομένως, η γραφή, ο έλεγχος και η επαλήθευση των προγραμμάτων
αυτών, ώστε να μπορέσουν να ενσωματωθούν ομαλώς στην σε καθημεριν εφαρμογές .


\section{Γλώσσες συγγραφής έξυπνων συμβολαίων}

Το θέμα της επιλογής κατάλληλης γλώσσας για τον προγραμματισμό έξυπνων
συμβολαίων έχει  απασχολήσει αρκετά τους ερευνητές στον τομέα ακόμα και πριν
την έλευση του blockchain. Στο κεφάλαιο \ref{dsls} θα δούμε και άλλα
παραδείγματα εφαρμογής του συναρτησιακού προγραμματισμού σε DSLs που
προορίζονται για περιγραφή οικονομικών συμβολαίων.

Η υλοποίηση των έξυπων συμβολαίων στο Bitcoin γίνεται δυνατή μέσω
της γλώσσας \textit{Bitcoin Script}, η εκτελέση της οποίας
γίνεται με τον χειρισμό μίας δομής στοίβας. Κάθε συναλλαγή που
εκτελεί ο τελικός χρήστης, μεταφράζεται σε μια σειρά εντολών στην
γλώσσα Bitcoin Script.

Η γλώσσα προγραμματισμού του Ethereum τρέχει πάνω από την εικονική μηχανή  EVM
\cite{ethereum}. Η γλώσσα αυτή, όπως και αυτή που θα εξετάσουμε στα κεφάλαια
\ref{chap:chapter4}. \ref{chap:chapter5} είναι αρκετά χαμηλού επιπέδου, και δεν
είναι σχεδιασμένη για να γράφεται ή να διαβάζεται από τον προγραμματιστή, αλλά
να αποτελεί την ``assembly-type'' γλώσσα που θα γράφεται στο blockchain,
διαθέσιμη για εξέταση αν αυτό χρειαστεί.

Η ύπαρξη μίας εικονικής μηχανής για την εκτέλεση έξυπνων συμβολαίων δίνει την
δυνατότητα για επαλήθευση των προγραμμάτων που εκτελούνται σε αυτές. Αυτό
μπορεί να γίνει με το verification αυτής της εικονικής μηχανής, που μπορεί να
γίνει με τεχνικές \emph{τυπικών μεθόδων} (formal methods). Εργαλεία απόδειξης
ορθότητας προγραμμάτων που τρέχουν στο EVM, καθώς και η διατύπωση ολοκληρωμένων
semantics για την εικονική μηχανή υπάρχουν στην βιβλιογραφία
(\cite{evmverification}, \cite{kevmverification}).

Κατά τον προγραμματισμό έξυπνων συμβολαίων όμως, όλες οι πλατφόρμες
υποστηρίζουν γλώσσες πιο υψηλού επιπέδου, για χρήση από τον προγραμματιστή. Για
παράδειγμα, στο Ethereum η πιο δημοφιλής γλώσσα προγραμματισμού συμβολαίων
είναι η γλώσσα Solidity \cite{solidity}, επηρεασμένη κυρίως από την γλώσσα
Javascript, και μεταγλωττίζεται σε κώδικα της εικονικής μηχανής EVM.

Η χρήση γλωσσών υψηλότερου επιπέδου για προγραμματισμό συμβολαίων, εκτός από
την ευκολία που παρέχει, ανοίγει ένα μέτωπο ευπαθειών προς εκμετάλλευση από
κακόβουλους παίκτες.


\section{Συναρτησιακές γλώσσες έξυπνων συμβολαίων} \label{dsls}

Η ύπαρξη αυτών των ευπαθειών κάνει την επιλογή της γλώσσας προγραμματισμού
συμβολαίων πολύ σημαντική. Οι συναρτησιακές γλώσσες έχουν συζητηθεί και
ερευνηθεί αρκετά στην κοινότητα των γλωσσών προγραμματισμού και φημίζονται για
την ασφάλειά και την χρησιμότητά τους για την κατασκευή μαθηματικά ορθών
προγραμμάτων. Ένα πρόγραμμα γραμμένο σε μία συναρτησιακή γλώσσα επιδίδεται
αρκετά πιο εύκολα σε τυπική ανάλυση και απόδειξη ιδιοτήτων σχετικά με την
λειτουργία του, και μπορεί να εκφραστεί πιο εύκολα με μαθηματικό συμβολισμό από
αντίστοιχα προγράμματα σε προστακτικές γλώσσες.  Η ύπαρξη του συστήματος τύπων
μπορεί να εντοπίσει και να εξαλείψει πολλές κατηγορίες λαθών, ήδη κατά την
μεταγλώττιση. Η πλατφόρμα έξυπνων συμβολα Τα χαρακτηριστικά αυτά κάνουν τις
συναρτησιακές γλώσσες δελεαστική επιλογή για τον προγραμματισμό έξυπνων
συμβολαίων.

Όπως αναφέρθηκε προηγουμένως,  η γλώσσα που προτείνει ο Szabo για την διατύπωση
machine-readable συμβολαίων είναι άμεσα επηρεασμένη από το έργο των S.P.Jones
et.al \cite{composingcontracts}  και \cite{howtowriteacontract}, όπου γίνεται
χρήση μίας συναρτησιακής DSL (domain specific language) για την περιγραφή
κλασσικών χρηματοoικονομικών προιόντων. Συγκεκριμένα, η γλώσσα που
χρησιμοποιούν είναι ευέλικτη και συνθέσιμη, χρησιμοποιώντας απλά συστατικά
στοιχεία τα οποία συνδυάζει για να κατασκευάσει πιο σύνθετα συμβόλαια.

Αρκετά συνηθισμένη επίσης είναι η χρήση τεχνικών τυπικής επαλήθευσης (formal
verification) για την εξάλειψη λαθών κατά τον προγραμματισμό έξυπνων
συμβολαίων. Όπως αναφέρθηκε παραπάνω, μπορούν να χρησιμοποιηθούν για την
απόδειξη ορθότητας προγραμμάτων που στοχεύουν την εικονική μηχανή, ή για
προγράμματα γλωσσών υψηλότερου επιπέδου.  Οι τεχνικές τυπικής επαλήθευσης είναι
παραδοσιακά ακριβές και δύσκολο να εφαρμοστούν σε μεγάλη κλίμακα σε πραγματικά
έργα λογισμικού, για αυτό βρίσκουν εφαρμογή κυρίως σε έργα λογισμικού όπου τα
λάθη στοιχίζουν ακριβά, όπως στον τομέα της αεροναυπηγικής και της κατασκευής
μικροεπεξεργαστών.

Κοντά στην Plutus Core, το θεωρητικό μοντέλο της οποίας θα εξετάσουμε στην
συνέχεια, είναι η γλώσσα Simplicity \cite{simplicity}, μια γλώσσα
\emph{συνδέσμων} (combinator-based), συνοδευόμενη με μία αφηρημένη μηχανή που
ορίζει την λειτουργική σημασιολογία της. Όπως και η Plutus Core, η Simplicity
έχει ελεγχθεί υπολογιστικά για την ορθότητά της με την βοήθεια προγραμμάτων
αποδείξεων. Αντίθετα από την Plutus Core όμως, δεν είναι Turing-complete.

Επιπλέον, ενώ είναι αρκετά εύκολη η χρήση προχωρημένων τεχνικών συναρτησιακού
προγραμματισμού για την συγγραφή προγραμμάτων σε Plutus Core καθώς βασίζεται
απευθείας στον λάμδα λογισμό, και συγκεκριμένα στην \FOM{}, δεν ισχύει το ίδιο
για την Simplicity.

Μία ακόμα συναρτησιακή πλατφόρμα συμβολαίων, η Tezos, εισάγει την γλώσσα
Michelson, ως έναν χαμηλού επιπέδου συνδυασμό της Forth με Lisp, υποστηρίζοντας
ισχυρούς τύπους.  Η σημασιολογία της Michelson έχει αποδειχθεί σωστή, και όπως
και η πλατφόρμα Plutus, υποστηρίζει προγραμματισμό συμβολαίων από τον χρήστη σε
γλώσσα υψηλότερου επιπέδου.

Οι παραπάνω γλώσσες καλύπτουν αποκλειστικά το on-chain κομμάτι των συμβολαίων,
δηλαδή το χαμηλού επιπέδου bytecode που εν τέλει θα καταλήξει να κατοικεί στο
blockchain. Στην αρχιτεκτονική Plutus, όπως θα συζητηθεί στο κεφάλαιο
\ref{subsec:plutus}, είναι σχεδιασμένη ώστε να αντιμετωπίζει ομοιόμορφα τόσο
τον on-chain, όσο και το κομμάτι του off-chain κώδικα, όπως το πορτοφόλι
(wallet) και το περιβάλλον με το οποίο έρχεται σε επαφή ο προγραμματιστής των
συμβολαίων (UI). Για να επιτύχει αυτόν τον σκοπό δεν επινοεί μια εντελώς
καινούργια γλώσσα, αλλά βασίζεται αρκετά σε ``γνωστές'' και παλιές τεχνολογίες
όπως το \FOM και η Haskell, και μπορεί να αξιοποιήσει το μεγάλο σώμα γνώσεων
γύρω από αυτές.

\subsection{Plutus και Plutus Core} \label{subsec:plutus}

Η πλατφορμα έξυπνων συμβολαίων Plutus βασίζεται πάνω στο Cardano blockchain.
Πρόκεται για ένα Proof-of-Stake blockchain πρωτόκολλο για επίτευξη
κατανεμημένης συμφωνίας (distributed consensus) \cite{ouroboros}.

Η αρχιτεκτονική Plutus έχει δύο κύρια συστατικά. Το πρώτο είναι ένα GHC plugin
που επιτρέπει στον προγραμματιστή να γράψει συμβόλαια σε Haskell.  Στη
συνέχεια, ο υψηλού επιπέδου κώδικας Haskell μεταγλωττίζεται στη γλώσσα Plutus
Core, μια χαμηλότερου επιπέδου γλώσσα, που προορίζεται να κατοικήσει στο
blockchain.

Ο προγραμματιστής μπορεί να γράφει μαζί τον κώδικα που εκτελείται τοπικά
(off-chain κώδικας) μαζί με τον κώδικα που ανεβαίνει στο blockchain (on-chain
κώδικας) σε Haskell, με τον on-chain κώδικα να μεταγλωττίζεται στην ενδιάμεση
αναπαράσταση \FIR{},  η οποία στη συνέχεια μεταγλωττίζεται σε Plutus Core.

Ο κώδικας Plutus Core, όπως και ο GHC Core, είναι επεκτάσεις του \FOM{}, του
πολυμορφικού λ-λογισμού.  Ο GHC Core είναι πιο πλούσια επέκταση και υποστηρίζει
αμοιβαία αναδρομικά bindings, αλγεβρικούς τύπους δεδομένων, εκφράσεις case,
μετατροπές τύπων (coercions) μεταξύ άλλων.

Αντίθετα, ο κώδικας Plutus Core παραμένει απλούστερος ως προς τα δομικά
χαρακτηριστικά που υποστηρίζει, προσπαθώντας να μείνει κοντά στην μαθηματική
μορφή του υποκείμενου λογισμού.  Όλα αυτά τα επιπλέον στοιχεία και
χαρακτηριστικά του GHC Core που προσθέτουν εκφραστικότητα μπορούν να
μεταφραστούν στον πιο ``μαθηματικό'' λογισμό αν η γλώσσα παρέχει έναν απλό τρόπο
έκφρασης της αναδρομής. Στα επόμενα κεφάλαια θα δειχθεί πως με χρήση των
στοιχειωδών εργαλείων που παρέχει η Plutus Core μπορούμε να εκφράσουμε
χαρακτηριστικά υψηλού επιπέδου.

Κύριο χαρακτηριστικό της πλατφόρμας Plutus είναι η παραγωγή του validator
script που θα καθορίσει τι σημαίνει ορθή εκτέλεση του συμβολαίου και τις
συνθήκες που πρέπει να επικρατούν για να ξοδέψει κάποιος τους πόρους του
συμβολαίου. Η ορθότητα του validator script είναι κεντρική σε κάθε blockchain,
καθώς μόλις ανέβει στο blockchain δεν μπορεί να τροποποιηθεί.

Μία γλώσσα για τέτοια χρήση οφείλει να είναι μικρή, συναρτησιακή, ώστε ο
προσδιορισμός της σημασιολογίας και της ανάλυσης της να απλοποιείται. Το
σύστημα του λ-λογισμού \FOM{} αποτελεί καλή αφετηρία για τους παραπάνω σκοπούς,
με μικρές αλλά ουσιαστικές τροποποιήσεις. Η γλώσσα δεν περιέχει απευθείας
τύπους δεδομένων και εκφράσεις case. Από κατασκευή, η \FOM{} περιέχει
παραμετροποιημένους τύπους , όπως η λιστα, (List A), όπου ο τύπος List είναι
μεγαλύτερης τάξης, συγκεκριμένα $\Type \rightarrow \Type$. Αρκετές ενδιάμεσες
γλώσσες υποστηρίζουν απευθείας τύπους δεδομένων, με τίμημα πιο σύνθετη
σημασιολογία. Οι τύποι δεδομένων υποστηρίζονται επομένως μέσω της κωδικοποίησης
Scott, που θα συζητηθεί στο κεφάλαιο \ref{chap:chapter3}.

Η Plutus Core δεν προορίζεται για μεταγλώττιση σε κώδικα μηχανής, και λόγω του
πεδίου χρήσης της, το μεγαλύτερο ποσοστό του χρόνου κατά την εκτέλεσή της
αφιερώνεται σε κρυπτογραφικές λειτουργίες. Το γεγονός αυτό καθιστά το επιπλέον
κόστος της κωδικοποίησης, σε αντίθεση με την απευθείας υποστήριξη τύπων
δεδομένων, έναν αποδεκτό συμβιβασμό.

\subsection{Marlowe}

Η παραπάνω προσέγγιση στην σχεδίαση μίας γλώσσας \emph{ειδικού-σκοπού}  (Domain
Specific Language) για προγραμματισμό συμβολαίων είναι η πρώτη ιστορικά
συνάντηση του συναρτησιακού προγραμματισμού και του κόσμου των ηλεκτρονικών
συμβολαίων.  Οι ιδέες αυτές έχουν επηρεάσει αρκετά τον σχεδιασμό της γλώσσας
Marlowe \cite{marlowe}, που έχει σχεδιαστεί με γνώμονα την χρήση της από
κάποιον που είναι ειδικός στην συγγραφή συμβολαίων, αλλά όχι στον
προγραμματισμό. Η χρήση μίας DSL για αυτόν τον σκοπό προσφέρει αρκετά
πλεονεκτήματα στους συγγραφείς των συμβολαίων:

Η Marlowe είναι μία γλώσσα ειδικού σκοπού ενσωματωμένη στην γλώσσα Haskell.
Προορίζεται για χρήση με την πλατφόρμα έξυπνων συμβολαίων Plutus.

Ο κώδικας Marlowe γίνεται συμβατός με την αρχιτεκτονική Plutus μέσω ενός
ερμηνευτή, καθώς η συγκεκριμένη προσέγγιση προσφέρει διάφορα σχεδιαστικά
πλεονεκτήματα όπως η δυνατότητα επαναχρησιμοποίησης της ίδιας υλοποίησης τόσο
στον on-chain, όσο και στον off-chain κώδικα, και η στενή σχέση με την
σημασιολογία της γλώσσας Marlowe.

Μπορούμε να είμαστε σίγουροι ότι κάποιοι τύποι ``κακών'' προγραμμάτων δεν
μπορούν να εκφραστούν στην γλώσσα. Με αυτόν τον τρόπο εξαλείφονται παθολογικές
συμπεριφορές που είναι δυνατό να εμφανιστούν σε γλώσσες γενικού σκοπού, όπως
π.χ C++, Javascript.

Για τα προγράμματα που κατασκευάζονται στην γλώσσα είναι πιο εύκολο να
επαληθευτεί ότι ικανοποιούν ορισμένες ιδιότητες. Για παράδειγμα, μπορούμε να
είμαστε σίγουροι ότι το συμβόλαιο θα εκτελέσει κάθε πληρωμή που είναι
προγραμματισμένο να εκτελέσει, ή ότι δεν θα βρεθεί ποτέ σε συγκεκριμένη ροή
εκτέλεσης. Κατά αυτόν τον τρόπο αυξάνουμε το επίπεδο εμπιστοσύνης που έχουμε
στο συμβόλαιο, και γινόμαστε πιο σίγουροι ότι εκτελεί όντως τις λειτουργίες που
είναι προγραμματισμένο να εκτελέσει και τίποτα περισσότερο ή λιγότερο.

Καθώς η Marlowe είναι γλώσσα ειδικού σκοπού, είναι αρκετά εύκολη η σχεδίαση
εργαλείων για την προσομοίωση των προγραμμάτων. Συνυπολογίζοντας ότι πρόκειται
για συναρτησιακή γλώσσα, γίνεται εύκολο να γραφτούν προσομοιωτές και ερμηνευτές
της γλώσσας με στόχο την εξαντλητική εξέταση και προσομοίωση των έξυπνων
συμβολαίων που γράφονται. Ο ερμηνευτής αυτός μετατρέπει τον κώδικα Marlowe σε
Plutus Core, έτοιμο για εκτέλεση μόλις του δοθεί η κατάλληλη είσοδος.

Γίνεται επομένως δυνατόν μέσω της Marlowe να γραφτούν σύνθετα συμβόλαια, όπως
συλλογικής χρηματοδότησης (crowdfunding), εγγύησης (escrow) ακόμα και
χρηματοοικονομικών παραγώγων.

