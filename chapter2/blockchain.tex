\section{Blockchain}

Η τεχνολογία του blockchain παρουσιάστηκε για πρώτη φορά από τον Satoshi Nakamoto το 2008 
\cite{bitcoin}, και αποτελεί την πρώτη υλοποίηση ψηφιακού νομίσματος, του Bitcoin που λύνει αποκεντροποιημένα το πρόβλημα του διπλοξοδέματος (double spending). Αποτελεί ένα μέσο επίλυσης χρηματικών συναλλαγών χωρίς την επέμβαση κάποιας κεντρικής αρχής, που βασίζεται σε κρυπτογραφικές μεθόδους για την επίτευξη της συνέναισης των συμμετεχόντων.

Σύντομα μετά την εν λόγω δημοσίευση που περιγράφει τις αρχές λειτουργίας του bitcoin και εισάγει
την έννοια του blockchain, δημοσιέυθηκε ως λογισμικό ανοιχτού κώδικα η υλοποίηση του συστήματος
σε C++ το 2009, πάλι από τον Satoshi Nakamoto και την ομάδα του, ο οποίος λίγο αργότερα εξαφανίστηκε
χωρίς να  γνωρίζουμε την πραγματική του ταυτότητα. μέχρι και σήμερα. 


Η δημιουργία ενός ψηφιακού νομίσματος ελεύθερο από τον έλεγχο μιας κεντρικής αρχής απασχολούσε αρκετά χρόνια την κοινότητα των cypherpunks την δεκαετία του 90', όπου μπήκαν τα θεμέλια για την
εισαγωγή της τεχνολογίας του blockchain από τον Nakamoto, όπως η χρήση των Merkle trees και των ψηφιακών υπογραφών.

Απο τότε, η τεχνολογία του blockchain έχει προσελκύσει μεγάλο ενδιαφέρον λόγω των εφαρμογών της και
θεωρείται από τις πιο πρωτοποριακές εφευρέσεις του 21ου αιώνα. 


\subsection{ Smart Contracts}

Η έννοια των έξυπων συμβολαίων (smart contracts), προηγείται αυτής του Bitcoin και μπορεί να εντοπιστεί 
στο 1997, από τον Nick Szabo \cite{szabo1}, ο οποίος οραματίστηκε αρκετές από τις σημερινές εφαρμογές
των έξυπνων συμβολαίων, όπως η εκκαθάριση πληρωμών, η διαχείριση και μεταφορά ιδιοκτησίας,  η 
εμπιστοσύνη σε ανώνυμα και ψευδώνυμα δίκτυα, και η αποκεντροποιημένη εκδοχή κλασσικών χρηματοοικονομικών συμβολαίων όπως τα παράγωγα. Προτείνει ακόμα μία γλώσσα για χρήση στον 
προγραμματισμό έξυπνων συμβολαίων, βαθειά επηρεασμένη από τον συναρτησιακό προγραμματισμό.
Το θέμα της επιλογής κατάλληλης γλώσσας για τον προγραμματισμό έξυπνων συμβολαίων έχει  απασχολήσει
 αρκετά τους ερευνητές στον τομέα ακόμα και πριν την έλευση του blockchain. Στο κεφάλαιο \ref{dsls} θα
   δούμε και άλλα παραδείγματα εφαρμογής του συναρτησιακού προγραμματισμού σε DSLs που προορίζονται
   για περιγραφή οικονομικών συμβολαίων.

   Αρκετά ενδιαφέρον είναι το γεγονός ότι η γλώσσα που προτείνει ο Szabo για την διατύπωση machine-
   readable συμβολαίων είναι άμεσα επηρεασμένη από το έργο των S.P.Jones et.al \cite{composingcontracts}
    και \cite{howtowriteacontract}, όπου γίνεται χρήση μίας συναρτησιακής DSL (domain specific language) για την περιγραφή κλασσικών χρηματοoικονομικών προιόντων. Συγκεκριμένα, η γλώσσα που χρησιμοποιούν
    είναι ευέλικτη και συνθέσιμη, χρησιμοποιώντας απλά συστατικά στοιχεία τα οποία συνδυάζει για να κατασκευάσει πιο σύνθετα συμβόλαια.

    Η παραπάνω σύνδεση του κόσμου των συναρτησιακής γλωσσών και τον domain specific languages 


     Η παραπάνω προσέγγιση στην σχεδίαση μίας DSL για προγραμματισμό συμβολαίων είναι η πρώτη ιστορικά
      συνάντηση του συναρτησιακού προγραμματισμού και του κόσμου των, βρίσκεται και πίσω 
        από την γλώσσα Marlowe \cite{Marlowe},

              
              Η υλοποίηση των έξυπων συμβολαίων στο Bitcoin γίνεται δυνατή μέσω της γλώσσας 
              \textit{Bitcoin Script}, η εκτελέση της οποίας γίνεται με τoν χειρισμό μίας δομής στοίβας. Κάθε
              συναλλαγή που εκτελεί ο τελικός χρήστης, μεταφράζεται σε μια σειρά εντολών στην γλώσσα
              Bitcoin Script. 


\section{Επιθέσεις σε έξυπνα συμβόλαια} 


\section{Συναρτησιακές γλώσσες έξυπνων συμβολαίων} \label{dsls}


Όπως αναφέρθηκε προηγουμένως,  η γλώσσα που προτείνει ο Szabo για την διατύπωση machine-readable συμβολαίων είναι άμεσα επηρεασμένη από το έργο των S.P.Jones et.al 
\cite{composingcontracts}  και \cite{howtowriteacontract}, όπου γίνεται χρήση μίας συναρτησιακής DSL
 (domain specific language) για την περιγραφή κλασσικών χρηματοoικονομικών προιόντων. Συγκεκριμένα, η
   γλώσσα που χρησιμοποιούν είναι ευέλικτη και συνθέσιμη, χρησιμοποιώντας απλά συστατικά στοιχεία τα
     οποία συνδυάζει για να κατασκευάσει πιο σύνθετα συμβόλαια. 
       
         Είναι λοιπόν φανερό ότι τα οφέλη που προσφέρει ο συναρτησιακός προγραμματισμός, δηλαδή η 
           συνθεσιμότητα, δυνατότητα για εύκολη τυπική ανάλυση και μετασχηματισμό, και η ασφάλεια του
             συστήματος τύπων, αποτελεί άκρως επιθυμητό χαρακτηριστικό για μία γλώσσα προγραμματισμού
               έξυπνων συμβολαίων.
                 
                  Αρκετά συνηθισμένη επίσης είναι η χρήση τεχνικών τυπικής επαλήθευσης (formal verification) για
                    την εξάλειψη λαθών κατά τον προγραμματισμό έξυπνων συμβολαίων. Οι τεχνικές τυπικής επαλήθευσης
                      είναι παραδοσιακά ακριβές και δύσκολο να εφαρμοστούν σε μεγάλη κλίμακα σε πραγματικά έργα
                        λογισμικού, για αυτό βρίσκουν εφαρμογή κυρίως σε έργα λογισμικού όπου τα λάθη στοιχίζουν ακριβά,
                          όπως στον τομέα της αεροναυπηγικής και της κατασκευής μικροεπεξεργαστών.
                            


                            \subsection{Plutus}
                             \label{subsec:plutus}

                             Η πλατφορμα έξυπνων συμβολαίων Plutus βασίζεται πάνω στο Cardano blockchain. Πρόκεται για ένα
                             Proof-of-Stake blockchain πρωτόκολλο για επίτευξη κατανεμημένης συμφωνίας (distributed consensus) \cite{ouroboros}. 

                             Η αρχιτεκτονική Plutus έχει δύο κύρια συστατικά. Το πρώτο είναι ένα GHC plugin που επιτρέπει στον
                             προγραμματιστή να γράψει συμβόλαια σε Haskell.  Στη συνέχεια, ο υψηλού επιπέδου κώδικας Haskell
                             μεταγλωττίζεται στη γλώσσα Plutus Core, μια χαμηλότερου επιπέδου γλώσσα, που προορίζεται να
                             κατοικήσει στο blockchain.


                              γλώσσα που μπορεί να χρησιμοποιηθεί για τον προγραμματισμό έξυπνων συμβολαίων. Ο προγραμματιστής 
                               μπορεί να γράφει μαζί τον κώδικα που εκτελείται τοπικά (off-chain κώδικας) μαζί με τον κώδικα που 
                                ανεβαίνει στο blockchain (on-chain κώδικας) σε Haskell, με τον on-chain κώδικα να μεταγλωττίζεται στην 
                                 ενδιάμεση αναπαράσταση \FIR{},  η οποία
                                  στη συνέχεια μεταγλωττίζεται σε Plutus Core.

