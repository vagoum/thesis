
Η μεταγλώττιση των μη-αναδρομικών τύπων δεδομένων είναι αρκετά απλή.  Η
στρατηγική μεταγλώττισης είναι συγκεντρωμένη στο σχήμα
\ref{fig:compile-datatypes}.  Χρησιμοποιεί τις βοηθητικές συναρτήσεις του
προηγούμενου κεφαλαίου (\ref{fig:fir_aux}) και ακολουθεί την κωδικοποίηση Scott
(\ref{sec:data-encoding} ). Στη συνέχεια της παραγράφου, ως παράδειγμα της
μεθόδου θα μετατραπέι η \FIR{} δήλωση του $\Maybe$ σε καθαρή \FOMF{}.


\begin{displaymath} d \defeq \datatype{\Maybe}{A}{\Match}{(\Nothing (), \Just
(A))} \end{displaymath} \begin{itemize} \item $\branchTy{c}{R}$ υπολογίζει τον
      τύπο του constructor του συγκεκριμένου branch, δηλαδή τον τύπο που
      παίρνει τα ορίσματα του αντίστοιχου branch, και επιστρέφει έναν τύπο
      εξόδου $R$.
  \begin{align*} \branchTy{\Nothing ()}{R} &= R\\ \branchTy{\Just A}{R} &= A
  \rightarrow R \end{align*}

\item $\dataKind{d}$ υπολογίζει το kind του datatype. Πρόκεται για το kind
  arrow μεταξύ όλων των παραμέτρων που καταλήγει στο $\Type$.
  $$\dataKind{\Maybe} = \Type \kindArrow \Type$$ \item $\scottTy{d}$ υπολογίζει
    τον Scott τύπο του datatype. Δεσμεύει τα ονόματα των παραμέτρων και
    υλοποιεί το ταίριασμα προτύπων χρησιμοποιώντας τους τύπους των branch.
  $$\scottTy{d} = \lambda A . \forall R . R \rightarrow (A \rightarrow R)
\rightarrow R$$ \item $\constrTy{c}{T}$ υπολογίζει τον τύπο ενός constructor
  $d$.  \begin{align*} \constrTy{\Nothing ()}{\Maybe} &= \forall A . \Maybe A\\
  \constrTy{\Just A}{\Maybe} &= \forall A . A \rightarrow \Maybe A \end{align*}
\item $\constr{d}{c}$ υπολογίζει τον constructor στο επίπεδο των όρων. Ο τύπος
  $R$ χρησιμοποιείται στη θέση του αφηρημένου τύπου δεδομένων, αφού έχουν
  ``δεσμευτεί'' μέσω type abstraction τα ορίσματά του. Στη συνέχεια κατασκευάζει
  την συνάρτηση ταιριάσματος που παίρνει όλες τις επιλογές και εφαρμόζει το
  $i$-οστό branch στα κατάλληλα στα ορίσματα.  \begin{align*}
    \constr{d}{\Nothing ()} &= \Lambda A . \Lambda R . \lambda (b_1 : R) (b_2 :
    A \typeArrow R) . b_1\\ \constr{d}{\Just (A)} &= \Lambda A . \lambda (v :
    A) . \Lambda R . \lambda (b_1 : R) (b_2 : A \typeArrow R) . b_2\ v\\
    \constr{d}{\Nothing ()} &= \Lambda A . \Lambda R . \lambda (b_1 : R) (b_2 :
    A \typeArrow R) . b_1 \end{align*} \item $\matchTy{d}$ είναι ο τύπος της
      συνάρτησης ταιριάσματος $$\matchTy{d} = \forall A . \Maybe A \rightarrow
      (\forall R . R \rightarrow (A \rightarrow R) \rightarrow R)$$ \item
        $\match{d}$ δίνει τον ορισμό της συνάρτησης ταιριάσματος, που
        ουσιαστικά είναι η ταυτοτική συνάρτηση στον Scott τύπο.  $$\match{d} =
      \Lambda A . \lambda (v : \Maybe A) . v$$ \item $\unveil{d}{t}$
        αντικαθιστά τον αφηρημένο τύπο δεδομένων μέσα σε έναν όρο, με τον
        ``ωμό'' Scott τύπο.  $$\unveil{d}{t} = \subst{\Maybe}{\lambda A .
        \forall R . R \rightarrow (A \rightarrow R) \rightarrow R}{t}$$
\end{itemize}



\begin{figure}[!ht] \centering \begin{minipage}[t]{15cm} \centering
  \begin{displaymath} \begin{array}{l@{\ }l@{\ }l}
    \multicolumn{3}{l}{\textsf{}}\\ \multicolumn{3}{l}{d = \datatype{X}{(\seq{Y
    :: K})}{x}{(\seq{c})}} \\ \multicolumn{3}{l}{c = x(\seq{T})}\\ \\
    \multicolumn{3}{l}{\textsc{Βοηθητικές συναρτήσεις}}\\ \constr{d}{c} &=&
    \Lambda (\seq{Y::K}) .  \lambda (\seq{a : T}).  \Lambda R .  \lambda
    (\seq{b : \branchTy{c}{R}}) ~b_i ~ \seq{a}\\ &&\textbf{where}~c=c_i\\
    \constrs{d} &=& \seq{\constr{d}{c}}\\ \match{d} &=& \Lambda (\seq{Y::K}).
    \lambda (x : (\scottTy{d}\ \seq{Y}) . x\\ \unveil{d}{t} &=&
    \subst{X}{\scottTy{d}}{t} \\ \\ \multicolumn{3}{l}{\textsc{Συνάρτηση
    μεταγλώττισης}}\\ \multicolumn{3}{l}{\compiledata(\tlet d \tin t)}\\ &=&
    (\Lambda (\dataBind{d}) . \lambda (\constrBinds{d}) . \lambda
    (\matchBind{d}) . t)\\ &&\{\scottTy{d}\}\ \\
  &&\seq{\unveil{d}{\constrs{d}}}\\ &&\match{d}\\ \end{array} \end{displaymath}
\end{minipage} \caption{Μεταγλώττιση μη αναδρομικών τύπων δεδομένων}
\label{fig:compile-datatypes} \end{figure}

Η διαίσθηση πίσω από την μεταγλώττιση είναι απλή, με χρήση της αφαίρεσης τύπων
σε όρους ($\Lambda$-συναρτήσεις)δεσμεύουμε τα ονόματα του datatype, των
constructors και της συνάρτησης ταιριάσματος.

Οι αφηρημένοι τύποι δεδομένων μπορούν να εκφραστούν με την χρήση
\emph{υπαρξιακών} τύπων (existential types \cite{tapl}). Οι υπαρξιακοί τύποι
δεν αναφέρονται στην παρούσα διπλωματική, αλλά αποτελούν το δυικό ανάλογο των
καθολικών ποσοδεικτών. Κάθε υπαρξιακός τύπος μπορεί επομένως να εκφραστεί με
την χρήση των καθολικών ποσοδεικτώνν, και η χρήση του $\unveil{d}{t}$ στους
κατασκευαστές του αφηρημένου τύπου εξυπηρετεί αυτήν την λειτουργία.

 Όπως είδαμε και στην αρχή του κεφαλαίου, ενώ κατά την μετάφραση των όρων
 $\tlet$ δημιουργούμε μία αφαίρεση στο επίπεδο τύπων ή όρων που εφαρμόζεται
 άμεσα, εδώ εναλλάσσουμε το abstraction με την εφαρμογή. Αυτό γίνεται γιατί ο
 τύπος δεδομένος πρέπει να παραμείνει abstract στους κατασκευαστές, ώστε να
 πάρουν τον κατάλληλο τύπο. Για αυτό χρειάζεται να αντικαταστήσουμε τον
 αφηρημένο τύπο, με την concrete μορφή του, ενέργεια που επιτελεί η συνάρτηση
 $\unveil{d}{t}$. Για αυτό και η χρήση ενός ακόμα abstraction/let δεν είναι
 σωστή, καθώς θα δημιουργούσαν ακόμα έναν αφηρημένο τύπο.

 Παρακάτω βλέπουμε συγκεντρωμένη την μεταγλώττιση του $\Maybe$ με χρήση των
 παραπάνω συναρτήσεων.

\begin{align*} &\compiledata(\tlet \datatype{\Maybe}{A}{\Match}{(\Nothing (),
  \Just (A))}\\ &\quad\quad \tin\ \Match\ \{\Int\}\ (\Just \{\Int\}1)\ 0\
  (\lambda x: \Int . x+1))\\ =\ &(\Lambda (\Maybe :: \Type \kindArrow \Type) .
  \tag{τύπους του $\Maybe$}\\ &\lambda (\Nothing : \forall A . \Maybe A) .
  \tag{τύπος του $\Nothing$}\\ &\lambda (\Just : \forall A . A \rightarrow
  \Maybe A) . \tag{τύπους του  $\Just$}\\ &\lambda (\Match : \forall A . \Maybe
  A \rightarrow \forall R . R \rightarrow (A \rightarrow R) \rightarrow R).
  \tag{τύπους του $\Match$}\\ &\Match\ \{\Int\}\ (\Just \{\Int\}1)\ 0\ (\lambda
  x: \Int . x+1)) \tag{body of the let}\\ &(\lambda A . \forall R . R
  \rightarrow (A \rightarrow R) \rightarrow R) \tag{ορισμός του $\Maybe$}\\
  &(\Lambda A . \Lambda R . \lambda (b_1 : R)\ (b_2 : A \rightarrow R) . b_1)
  \tag{ορισμός του $\Nothing$}\\ &(\Lambda A . \lambda (v_1 : A) . \Lambda R .
  \lambda (b_1 : R)\ (b_2 : A \rightarrow R) . b_2\ v_1) \tag{ορισμός του
  $\Just$}\\ &(\Lambda A . \lambda (v : \forall R . R \rightarrow (A
\rightarrow R) \rightarrow R) . v) \tag{ορισμός του $\Match$} \end{align*}

Από τα παραπάνω παρατηρούμε:

\begin{itemize} \item Ο κατασκευαστής $\Just$ πρέπει να δώσει τον αφηρημένο
      τύπο στο σώμα του $tlet$, αλλιώς η ερφαμοργή του $\Match$ δεν θα
      συνοδεύεται από τον σωστό τύπο.  \item O Scott τύπους παράγεται στον
      ορισμό $\Just$.  \item Η συνάρτηση ταιριάσματος $\Match$ μετατρέπει τον
        αφηρημένο τύπο στην ``ωμή'' Scott εκδοχή του. Υπολογιστικά είναι απλά η
ταυτοτική συνάρτηση.  \end{itemize}
