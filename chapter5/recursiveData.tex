

Στην παράγραφο αυτή θα εξεταστεί η μεταγλώτιση των αναδρομικών τύπων δεδομένων
από την \FIR{}  σε \FOMF. Το πέρασμα αυτό θα χρησιμοποιήσει την δυνατότητα μεταγλώτισης
των μη-αναδρομικών τύπων δεδομένων, όπως το είδαμε στο προηγούμενο κεφάλαιο. Η μεταγλώτιση
των αναδρομικών τύπων δεδομένων μαζί με βοηθητικές συναρτήσεις για ευκολότερη παρουσίαση που
είδαμε και στην περίπτωση των μη-αναδρομικών datatypes (\ref{fig:fir_aux}). 


Η υποστήριξη απλά αναδρομικών τύπων είναι αρκετά απλή. Αρκεί να εκφράσουμε τον αναδρομικό
τύπο ως type-level συνάρτηση, συγκεκριμένα ως pattern functor (\ref{subsec:grammar}), αντικαθιστώντας
κάθε αναδρομική εμφάνιση με ένα όρισμα που χρησιμοποιείται αναδρομικά στην συνάρτηση. Στη συνέχεια
εφαρμόζωντας τον τελεστή σταθερού σημείου παράγουμε τον τελικό datatype.

Η ύπαρξη γινομένων στο επίπεδο των kinds θα έκανε τετριμμένη την κωδικοποίηση. Στο κεφάλαιο \ref{chap:chapter4} όμως, είδαμε την γραμματική και τους κανόνες της \FOMF{} και \FIR{}, που δεν υποστηρίζουν
τέτοιου τύπου γινόμενα.

Επομένως θα χρειαστεί να \emph{κωδικοποιήσουμε} τα γινόμενα στο επίπεδο των τύπων με τις 
τεχνικές που έχουμε στην διάθεσή μας. Σε ένα περιβάλλον με ισχυρούς τύπους, το γινόμενο $n$ kinds
μπορεί να εκφραστεί ως συνάρτηση από έναν δείκτη σε μία τιμή. Ο δείκτης αναφέρεται στην θέση του
στοιχείου του γινομένου. Αντί για φυσικό αριθμό, θα χρησιμοποιήσουμε μία διαφορετική ετικέτα που
δέχεται παραμέτρους τύπων, ένα χαρακτηριστικό που έλειπε από την αρχική δουλειά \cite{fixmutualgeneric}
και συζητήθηκε μόνο με χρήση εξαρτημένων τύπων στο \cite{genericwithindexed}. Ο ``δείκτης" που χρησιμοποιείται.
 
 
Η μεταγλώττιση της αναδρομικής περίπτωσης στην \FIR{} είναι αρκετά κοντά με την μη-αναδρομική
που παρουσιάστηκε στην προηγούμενη παράγραφο. Η ειδοποιός διαφορά είναι ότι ``δένουμε" τα ονόματα
των constructors, των συναρτήσεων ταιριάσματος, των ονομάτων των τύπων και των ορισμάτων μαζί,
και κατασκευάζουμε τον τύπο της οικογένειας κάνοντας χρήση του πλήρους αναδρομικού τύπου, 
που γνωρίζει για όλα τα datatypes-μέλη της οικογένειας.
 

%We define the compilation scheme for non-recursive datatype bindings in
%\cref{fig:compile-recursive-datatypes}, along with a number of auxiliary
Θα γίνει χρήση των συναρτήσεων από το σχήμα
\cref{fig:compile-datatypes}, δίνοντας τις αναγκαίες παραλλαγές για την αναδρομική
περίπτωση όπου αυτό χρειάζεται.

\begin{figure}[!ht]
    \centering
    \begin{minipage}[t]{15cm}
      \centering
      \begin{displaymath}
      \begin{array}{l@{\ }l@{\ }l}
  \multicolumn{3}{l}{\textsf{}}\\
  \multicolumn{3}{l}{l = \tlet \rec \seq{d} \tin t} \\
  \multicolumn{3}{l}{d = \datatype{X}{(\seq{Y :: K})}{x}{(\seq{c})}} \\
  \multicolumn{3}{l}{c = x(\seq{T})}\\
  \\
  \multicolumn{3}{l}{\textsc{Χρήσιμες συναρτήσεις}}\\
  \tagKind{l}
  &=& \seq{\dataKind{d}} \kindArrow \Type\\
  \dtTag{l}{d}
  &=& \lambda (\seq{Y::K}) . \lambda (\seq{X :: \dataKind{d}}) . X_i\ \seq{Y}\\
  &&\textbf{where}~d=d_i\\
  \dtInst{f}{l}{d}
  &=& \lambda (\seq{Y::K}). f\ (\dtTag{l}{d}\ \seq{Y})\\
  \dtFamily{l}
  &=& \lambda (r :: \seq{\dataKind{d}} \kindArrow \Type)\ . \lambda (t :: \tagKind{l}) . 
  \tlet \seq{X = \dtInst{r}{l}{d}} \tin t\ \seq{\scottTy{d}}\\
  \dtInstFinal{l}{d}
  &=& \lambda (\seq{Y::K}) . \ifix \dtFamily{l}\ (\dtTag{l}{d}\ \seq{Y})\\
  \constrRec{l}{d}{c}
  &=&\Lambda (\seq{Y::K}) . 
  \lambda (\seq{a : T}) . 
  \wrap \dtFamily{l}\ (\dtTag{l}{d}\ \seq{Y})\
  (\Lambda R . 
  \lambda (\seq{b : \branchTy{c}{R}}) . 
  ~b_k ~ \seq{a})\\
  &&\textbf{where}~d=d_i, c=c_k\\
  \constrsRec{l}{d} &=& \seq{\constrRec{l}{d}{c}}\\
  \matchRec{l}{d}
  &=& \Lambda (\seq{Y::K}). \lambda (x : \dtInstFinal{l}{d}\ \seq{Y}) . \unwrap x\\
  \unveilRec{l}{t}
  &=& \subst{X_1}{\dtInstFinal{l}{1}}{\dots \subst{X_n}{\dtInstFinal{l}{d_n}}{t}} \\
  \\
  \multicolumn{3}{l}{\textsc{Συνάρτηση μεγαγλώττισης}}\\
  \compiledatarec(l)
  &=&(\Lambda (\seq{\dataBind{d}}) . \lambda (\seq{\constrBinds{d}}) . \lambda (\seq{\matchBind{d}}) . t)\\
  &&\{ \seq{\dtInstFinal{l}{d}} \} \\
  &&\seq{\unveilRec{l}{\constrsRec{l}{d}}}\\
  &&\seq{\matchRec{l}{d}}
    \end{array}
    \end{displaymath}
    \end{minipage}
        
    \caption{Μεταγλώττιση αναδρομικών τύπων δεδομένων}
    \label{fig:compile-recursive-datatypes}
\end{figure}


Στην παράγραφο που ακολουθεί θα παρουσιαστεί βήμα-βήμα
η μεταγλώττιση των αμοιβαία αναδρομικών ορισμών των τύπων δεδομένων
$\Tree$ και $\Forest$. 

\begin{align*}
d_1 &\defeq \datatype{\Tree}{A}{\textsf{matchTree}}{(\Node (A, \Forest A))}\\
d_2 &\defeq \datatype{\Forest}{A}{\textsf{matchForest}}{(\NNil(), \CCons(\Tree A, \Forest A))}
\end{align*}
\begin{itemize}
  \item $\tagKind{l}$ εκφράζει το kind των datatypes της οικογένειας $l$.. Η διάκριση μεταξύ των περιπτώσεων
  των μελών της οικογένειας γίνεται μέσω της δομής του tag. Η χρήση tags για κωδικοποίηση τύπων
  δεδομένων είναι γνωστή από την βιβλιγραφία γύρω από τον \emph{γενικευμένο προγραμματισμό} (generic
  programming). Τα tags που θα χρησιμοποιήσουμε μιμούνται την κωδικοποίηση Scott των ν-άδων (tuples).
  Η $\tagKind{l}$ συγκεντρώνει τα kinds των διαφορετικών περιπτώσεων.
    $$\tagKind{l} = (\Type \kindArrow \Type) \kindArrow (\Type \kindArrow \Type) \kindArrow \Type$$
  \item $\dtTag{l}{d}$ δίνει τον ορισμό του τύπου του tag για κάθε τύπο δεδομένων $d$ της οικογένειας.
  Τα tags ζούν στο επίπεδο των τύπων, και δέχονται σαν ορίσματα τις παραμέτρους των datatypes καθώς
  και συναρτήσεις που αντιστοιχούν στην κάθε περίπτωση, και το εκάστοτε tag εφαρμόζει τις παραμέτρους
  στην σωστή περίπτωση. Έτσι υποστηρίζονται οι παραμετροποιημένοι datatypes, σημείο που υστερούσε
  από την βιβλιογραφία και έχει εξερευνηθεί κυρίως στο \cite{genericwithindexed}. Εδώ αντιθέτως με την
  προηγούμενη δουλειά δεν χρησιμοποιούμε περιβάλλον με \emph{εξαρτημένους τύπους} (dependent types).
  
    \begin{align*}
    \dtTag{l}{\Tree} &= \lambda A . \lambda (v_1 :: \Type \kindArrow \Type) (v_2 :: \Type \kindArrow \Type) . v_1\ A\\
    \dtTag{l}{\Forest} &= \lambda A . \lambda (v_1 :: \Type \kindArrow \Type) (v_2 :: \Type \kindArrow \Type) . v_2\ A
    \end{align*}
  \item Η βοηθητική συνάρτηση $\dtInst{f}{l}{d}$ εφαρμόζει τον τύπο $f$ στο tag του 
    datatype $d$ της οικογένειας. Ο τύπος $f$ για παράδειγμα θα μπορούσε να αναφέρεται στον
    τύπο $TreeForest$, της αμοιβαία αναδρομικής οικογένειας, με τους τύπους-μέλη $\Tree$ και $\Forest$.
    Η εφαρμογή εδώ αντιστοιχεί στο ταίριασμα προτύπων, η έκφραση της μορφής $\dtInst{f}{l}{\Tree}$ 
    αναφέρεται στην περίπτωση $f$ του τύπου $\Tree$.
    \begin{align*}
    \dtInst{f}{l}{\Tree} &= \lambda A . f\ (\dtTag{\seq{d}}{\Tree}\ A)\\
    \dtInst{f}{l}{\Forest} &= \lambda A . f\ (\dtTag{\seq{d}}{\Forest}\ A)
    \end{align*}
  \item Η συνάρτηση $\dtFamily{l}$ ορίζει τον τύπο της αμοιβαία αναδρομικής οικογένειας. Πρόκειται για μία συνάρτηση
   που δέχεται δύο ορίσματα, το αναδρομικό που θα χρησιμοποιηθεί από τον τελεστή $\ifix$ για να ``δέσει
   τον κόμπο" της αναδρομής, και το tag που δηλώνει την περίπτωση του τύπου. Στη συνέχεια εφαρμόζει 
   στο tag τους κατά Scott κωδικοποιημένους τύπους της οικογένειας, οι οποίοι δηλώνονται στο $\tlet$ και 
   έχουν με την σειρά τους αρχικοποιηθεί από το αναδρομικό όρισμα.
  
    \begin{align*}
    \dtFamily{l} =&\ \lambda r\ t . \tlet \\
        &\quad\Tree = \dtInst{r}{l}{\Tree}\\
        &\quad\Forest = \dtInst{r}{l}{\Forest}\\
      &\tin t\ \scottTy{d_1}\ \scottTy{d_2}\\
    \scottTy{d_1} =&\ \lambda A . \forall R . (A \rightarrow \Forest A \rightarrow R) \rightarrow R\\
    \scottTy{d_2} =&\ \lambda A . \forall R . R \rightarrow (\Tree A \rightarrow \Forest A \rightarrow R) \rightarrow R
    \end{align*}
  \item Στην $\dtInstFinal{l}{d}$ χρησιμοποιείται για την αρχικοποίηση της οικογένειας ο πλήρης
  αναδρομικός τύπος, δηλαδή η εφαρμογή του $\ifix$ στον $\dtFamily{l}$ που ορίστηκε προηγουμένως.
    $$\dtInstFinal{l}{\Tree} = \lambda A . \ifix \dtFamily{l}\ (\dtTag{l}{\Tree}\ A)$$
  \item Η συνάρτηση $\constrRec{l}{d}{c}$ κατασκευάζει τον constructor $c$ του τύπου δεδομένων $d$
  της οικογένειας. Έχουμε την ίδια συμπεριφορά με την μη-αναδρομική περίπτωση, εδώ όμως γίνται
  χρήση του $\wrap$ που πακετάρει τα ορίσματα και τον τύπο αποτελέσματος και επιστρέφει το
  κατάλληλο κλαδί (branch).
    \begin{align*}
    \constrRec{l}{\Tree}{\Node} =&\ \Lambda A . \lambda (v_1 : A) (v_2 : \Forest A) .\\
                               &\wrap\ \dtInstFinal{l}{\Tree}\ A\\
                               &(\Lambda R . \lambda (b_1 : A \rightarrow \Forest A \rightarrow R) . b_1\ v_1\ v_2)\\
    \constrRec{l}{\Forest}{\NNil} =&\ \Lambda A . \\
                               &\wrap\ \dtInstFinal{l}{\Forest}\ A\\
                               &(\Lambda R . \lambda (b_1 : R) (b_2 : \Tree A \rightarrow \Forest A \rightarrow R) . b_1)\\
      \constrRec{l}{\Forest}{\CCons} =&\ \Lambda A . \lambda (v_1 : \Tree A) (v_2 : \Forest A) . \\
                               &\wrap\ \dtInstFinal{l}{\Forest}\ A\\
                               &(\Lambda R . \lambda (b_1 : R) (b_2 : \Tree A \rightarrow \Forest A \rightarrow R) . b_2\ v_1\ v_2)
    \end{align*}
  \item Η $\matchRec{l}{d}$ μας δίνει τον τύπο της συνάρτησης ταιριάσματος για τον datatype $d$. Παρόμοια
  με την μη αναδρομική περίπτωση, η συνάρτηση ταιριάσματος είναι η ταυτοτική  συνάρτηση στον αφηρημένο τύπο δεδομένων (Algebraic Datatype). Το ίδιο συμβαίνει και στην αναδρομική περίπτωση, με την προσθήκη του  $\unwrap$ που ``αναλύει" τον ADT.
    \begin{align*}
    \matchRec{l}{\Tree} &= \Lambda A . \lambda (v : \Tree A) . \unwrap\ v\\
    \matchRec{l}{\Forest} &= \Lambda A . \lambda (v : \Forest A) . \unwrap\ v
    \end{align*}
  \item $\unveilRec{l}{t}$ ``ξετυλίγει'' τους datatypes όπως πριν αντικαθιστώντας τις
  εμφανίσεις των δεσμεύσεων των ονομάτων τους με τον κατα Scott κωδικοποιημένο τύπο.
\end{itemize}