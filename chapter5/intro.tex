Στο κεφάλαιο αυτό θα δούμε πως μπορούμε να μεταγλωττίσουμε τις υψηλού-επιπέδου
δηλώσεις τύπων δεδομένων που υποστηρίζει η \FIR{} στην \FOMF, εφαρμόζοντας
δηλαδή desugaring σε μία γλώσσα που έχει μελετηθεί θεωρητικά και έχουν
αποδειχτεί θεωρήματα ορθότητας για αυτή. Δεν θα αναφερθούμε στην μεταγλώττιση
στο επίπεδο των όρων, αλλά ούτε και στην μετάφραση αναδρομικών $\tlet$ τύπων,
καθώς η περίπτωση των τύπων δεδομένων είναι όχι μόνο πιο πλούσια, αλλά και
υπερσύνολο των αναδρομικών δηλώσεων τύπων.

Κατά την μετάφραση από \FIR{} σε \FOMF{}, θα φανεί χρήσιμο το ότι η πρώτη είναι
υπερσύνολο της δεύτερης, επομένως μπορούμε να ``απαλείψουμε'' κάθε στοιχείο της
\FIR{} σταδιακά, μέχρι να φτάσουμε σε καθαρή \FOMF{}. Για παράδειγμα στην
μεταγλώττιση των αναδρομικών datatypes θα χρησιμοποιήσουμε την μεταγλώττιση των
μη-αναδρομικών $\tlet$.

Καθώς θα ασχοληθούμε με τα bindings τύπων, και συγκεκριμένα τύπων δεδομένων δεν
εξετάζουμε ``ετερογενή'' $\tlet$ που εμπεριέχουν τόσο δηλώσεις όρων όσο και
τύπων. Από τους κανόνες γραμματικής της \FIR{} βλέπουμε ότι τέτοια δήλωση είναι
δυνατή. Το γεγονός αυτό δεν αποτελεί εμπόδιο στην περίπτωση μη αναδρομικών
δεσιμάτων, καθώς οι νέοι ορισμοί δεν μπορούν να βρεθούν στο δεξιό μέλος μίας
δήλωσης.

Βέβαια δεν μπορούμε να απαιτήσουμε ο χειρισμός των δεσιμάτων να είναι τόσο
απρόσκοπτος και στην αναδρομική περίπτωση, και για αυτόν τον λόγο θα θεωρήσουμε
\emph{ομογενή} $\tlet$, που εμπεριέχουν δηλώσεις μόνο μίας κατηγορίας, στην
παρούσα εργασία μόνο τύπων δεδομένων.

Επιπροσθέτως, δεν βρίσκεται στο άμεσο ενδιαφέρον μας η υποστήριξη αναδρομικών
τύπων σε $\tlet$ δηλώσεις, άρα οι τύποι δεν εξαρτώνται από datatypes ή
αντίστροφα. Η έλλειψη εξαρτημένων τύπων σημαίνει ότι οι τύποι δεν εξαρτώνται
από όρους άρα μπορούμε με ασφάλεια να χωρίσουμε τις κατηγορίες των
$\tlet$-bindings.
