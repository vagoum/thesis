\section{\FOM}
\label{sec:fom}

Η γλώσσα \FOM{} αποτελεί επέκταση του λ-λογισμού με απλούς τύπους. Αν θέλουμε να ορίσουμε
την θέση της στον λαμβδα κύβο, αποτελεί την προέκταση του λ-λογισμού με απλούς τύπους με
δύο χαρακτηριστικά, \emph{τελεστές τύπων} (\emph{type operators} ) και πολυμορφισμό.

Η \FOM{} ως έχει είναι ισχυρά κανονικοποιήσιμη (\emph{strongly normalizing} ), δηλαδή κάθε ακολουθία
αποτιμήσεων $t_1 \rightarrow \dots \rightarrow $ τερματίζει. Από το  γεγονός αυτό προκύπτει ότι ο έλεγχος
και η ανακατασκευή τύπων για την \FOM{} είναι decidable, αναμενόμενο, καθώς δεν έχουμε προσθέσει αναδρομή στην γλώσσα. Η αποτίμηση στο επίπεδο των τύπων πραγματοποιείται κατά το typechecking
της γλώσσας

\section{Ορισμός της \FOMF{} και της \FIR{}}
\label{sec:fomf}


Η \FIR{} αποτελεί μια προέκταση της \FOMF{}, που με τη σειρά της είναι επέκταση της \FOM{}.
Στα σχήματα που ακολουθούν βλέπουμε τους κανόνες σύνταξης (σχήμα \ref{fig:fir_syntax}), σύνθεσης και
ισοδυναμίας τύπων και kinds(σχήματα \ref{fig:fir_typing}, \ref{fig:fir_typeq} , \ref{fig:fir_kinding} αντίστοιχα),
καθώς και ορθής κατασκευής των bindings και των κατασκευαστών που υποστηρίζει η \FIR{} (σχήμα
\ref{fig:fir_wellformed}) . Οι μη-υπογραμμισμένες περιπτώσεις ανήκουν στην \FOM{}, και οι προσθήκες στις \fomfDiff{\FOMF{}} και \firDiff{\FIR{}} αντίστοιχα.

Σε όλα τα παρακάτω, χρησιμοποιούμε τον συμβολισμό $\seq{t}$ για να δηλώσουμε την ακολουθία
$t_1, \cdots, t_n$, καθώς και τις βοηθητικές συναρτήσεις στο σχήμα \ref{fig:fir_aux}. Οι συναρτήσεις
αυτές θα φανούν χρήσιμες στην έκφραση των κανόνων τύπων και ακόμα περισσότερο στην μεταγλώττιση
των binding τύπων δεδομένω από την \FIR{} σε καθαρή \FOMF{} στο επόμενο κεφάλαιο.

Στις παρακάτω παραγράφους θα σχολιαστούν λεπτά σημεία που αφορούν τον ορισμό της \FOMF{} και
FIR{} που αξίζουν ιδιαίτερη αναφορά.

\subsection{Γραμματική της \FOMF{} και \FIR{} }
\label{subsec:grammar}


Στο σχήμα \ref{fig:fir_syntax} φαίνεται η χρήση των όρων $\wrap$ και $\unwrap$ που αποτελούν
τους μάρτυτες του ισομορφισμού των αναδρομικών τύπων που υποστηρίζονται. Παρατηρούμε
πως το $\wrap$ της γλώσσας μας είναι πλήρως saturated, μια επιλογή που απλοποιεί την χρήση
του καθώς και τους κανόνες τύπων που δίνονται.

Το πρώτο όρισμα του $\wrap$ αντιστοιχεί στην δομή που ονομάζουμε \emph{pattern functor}, και περιγράφει
την δομή του αναδρομικού τύπου, που ουσιαστικά πρόκεται για συνάρτηση στο επίπεδο των τύπων.
Το δεύτερο όρισμα του $\wrap$ αποθηκεύει τις παραμέτρους του αναδρομικού τύπου δεδομένων
που περιγράφει ο pattern functor. Τέλος το τρίτο όρισμα περιέχει τον όρο που θέλουμε να τυλίξουμε.

Στην περίπτωση του $\unwrap$ δεν χρειαζόμαστε ως επιπλέον όρισμα τον αναδρομικό τύπο, ή τις παραμέτρους του, καθώς η πληροφορία για το πως πρέπει να ξετυλιχθεί ο όρος περιέχεται σε αυτόν,
και συγκεκριμένα στον τύπο του. Από την μορφή των κανόνων τύπων συμπεραίνουμε πως κάθε όρος
που μπορεί να χρησιμοποιηθεί σε έναν $\unwrap$ έχει ήδη ``τυλιχτεί" προηγουμένως με χρήση του
$\wrap$.

Για αυτό λοιπόν και στις τιμές της \FOMF{} προστίθενται και οι όροι $\wrap$, που μπορούν να υπάρξουν
ως τιμές αν παραμετροποιηθούν από τιμές-όρους. Οι όροι $\unwrap$ αντιθέτως δεν μπορούν να είναι values
καθώς πάντα μπορούμε να εφαρμόσουμε $\wrap$ σε έναν όρο $\unwrap$.

Ο τύπος του όρου είναι αυτός της συνάρτησης στο επίπεδο των τύπων, ακολουθώντας το παράδειγμα
στην παράγραφο \ref{sec:rectypes-intro}, όπου εκφράσαμε τον αναδρομικό τύπο $\List$ σε θεωρητικό
πλαίσιο χρησιμοποιώντας συναρτήσεις τύπων. Ο συμβολισμός προσεγγίζει αρκετά αυτόν της θεωρίας,
με μόνη διαφορά την χρήση του $\lambda$, αντί για $\mu$ που έχουμε στην θεωρία αναδρομικών τύπων.


\begin{figure}[!ht]
  \centering
  \begin{minipage}[t]{15cm}
  \centering
  \begin{displaymath}
  \begin{array}{lllll}
    \textrm{όροι}    & t, u   & ::= & x                           & \textrm{variable}\\
                      &        &     & \lambda x:T.t               & \textrm{lambda abstraction}\\
                      &        &     & t ~ t                       & \textrm{function application} \\
                      &        &     & \Lambda X :: K . t          & \textrm{type abstraction}\\
                      &        &     & t~\{T\}                      & \textrm{type application}\\
                      &        &     & \fomfDiff{\wrap  T ~U ~ t} & \textrm{wrap}\\
                      &        &     & \fomfDiff{\unwrap t}        & \textrm{unwrap}\\
                      &        &     & \firDiff{\tlet ~ [\rec] ~ \seq{b} ~ \tin ~ t} & \textrm{let}\\
                      &        &     &                             &   \\
    \textrm{bindings} & b      & ::= & \firDiff{x : T = t}         & \textrm{term binding}   \\
                      &        &     & \firDiff{X :: K = T}        & \textrm{type binding}\\
                      &        &     & \firDiff{\datatype{X}{(\seq{Y :: K})}{x}{\seq{c}}} & \textrm{datatype binding}  \\
                      &        &     &                             &   \\
    \textrm{κατασκευαστές} & c   & ::= & x~(\seq{T})                 & \\
    \textrm{τιμές}   & v      & ::= & \lambda x:T.t               & \textrm{lambda abstraction }   \\
                      &        &     & \Lambda X :: K.t            & \textrm{type abstraction}\\
                      &        &     & \fomfDiff{\wrap ~ T ~ U ~v} & \textrm{wrap}\\
                      &        &     &                             &   \\
    \textrm{τύποι}    & T,U    & ::= & X                           & \textrm{type variable}\\
                      &        &     & T \typeArrow U                      & \textrm{arrow type}\\
                      &        &     & \forall X :: K. T           & \textrm{universal type}\\
                      &        &     & \lambda X :: K. T           & \textrm{function type}\\
                      &        &     & T ~ U                       & \textrm{function application}\\
                      &        &     & \fomfDiff{\ifix ~ T ~ U}    & \textrm{fixpoint type}\\
                      &        &     &                             &   \\
    \textrm{πλαίσια} & \Gamma & ::= & \varnothing                 & \textrm{empty}\\
                      &        &     & \Gamma, x:T                 & \textrm{term variable binding}\\
                      &        &     & \Gamma, X::K                & \textrm{type variable binding}\\
                      &        &     &                             &    \\
    \textrm{kinds}     & K      & ::= & \Type                       & \textrm{type kind}\\
                      &        &     & K \kindArrow K              & \textrm{arrow kind}\\
  \end{array}
  \end{displaymath}
  \end{minipage}

  \caption{Σύνταξη και γραμματική της \FIR}
  \label{fig:fir_syntax}
\end{figure}


\subsection{Κανόνες και ισοδυναμία τύπων}

H \FOMF{} περιέχει το σύστημα του λ-λογισμού με απλούς τύπους στο επίπεδο των τύπων, καθώς υποστηρίζει
δημιουργία συνάρτησης (abstraction) και εφαρμογή (application), μεταξύ τύπων με το σωστό kind, βλέπε
σχήμα \ref{fig:fir_kinding}. Επομένως δημιουργείται υπολογισμός στο επίπεδο των τύπων και χρειάζεται
να γίνει αποτίμηση των τύπων στην τελική τους μορφή. Στο σχήμα \ref{fig:fir_typeq} παρουσιάζονται οι
κανόνες για το πότε δύο τύποι είναι ισοδύναμοι, προκύπτουν δηλαδή από την διαδικασία της αποτίμησης.

Η διαδικασία αυτή είναι ασφαλής καθώς η γλώσσα \FOMF{} είναι ισχυρά κανονικοποιήσιμη, όπως αναφέρθηκε
και στην παράγραφο \ref{sec:fom}. Ο υπολογισμός στο επίπεδο των τύπων τερματίζει πάντα και μας δίνει
έγκυρο τύπο.

Σε μία γλώσσα με αυτό το χαρακτηριστικό που προορίζεται για χρήση στο blockchain, είναι σημαντικό
οι τύποι να κανονικοποιούνται πριν αποθηκευτούν ως κώδικας στο blockchain, για μείωση χώρου αλλά
και gas κατά την εκτέλεση των προγραμμάτων.



\subsection{Τύποι και όροι παραμετροποιήσιμοι από τύπους}

Ως επέκταση της γλώσσας \FOM, υποστηρίζονται όροι παραμετροποιήσιμοι από τύπους.
(\emph{type abstraction}
Έτσι εκφράζεται ο πολυμορφισμός στην γλώσσα, οι πολυμορφικοί όροι παραμετροποιούνται
από μία μεταβλητή τύπου, που δίνεται σαν όρισμα όταν καλείται η συνάρτηση. Με παρόμοιο
τρόπο χειρίζεται ο πολυμορφισμός και στον GHC, τον compiler της Haskell. Οι συναρτήσεις
στο επίπεδο των όρων που παραμετροποιούνται από τύπους δημιουργούνται με την χρήση
του $\Lambda$. Στο κεφάλαιο \ref{chap:chapter4} θα γίνει έντονη χρήση των type abstractions
κατά την μεταγλώττιση των τύπων δεδομένων από την \FIR{} σε \FOMF{}.


Επομένως όταν
βλέπουμε κατασκευές της γλώσσας που περιέχουν $\Lambda$ ξέρουμε ότι ανήκουν
στο επίπεδο των \textbf{όρων}. Ο τύπος του όρου $ \Lambda X :: K . t $ είναι ο universally
quantified $(\forall X::K.T)$, όπως φαίνεται και από τον κανόνα T-TAbs του σχήματος \ref{fig:fir_typing}

Στην γραμματική της \FOMF{} βλέπουμε και την ύπαρξη τύπων που παραμετροποιούνται από
τύπους ( $\lambda X :: K. T)$ Τα type-level lambdas αυτά``κατοικούν" στο επίπεδο των τύπων,
και είναι εμφανές από το περιβάλλον που χρησιμοποιούνται σε ποια αναφερόμαστε. Η αποτίμηση
αυτών των συναρτήσεων περιλαμβάνεται στο σχήμα \ref{fig:fir_typeq}, στους κανόνες Q-Abs και
Q-Beta.


\newcommand{\gammaterm}{\Gamma^{\textrm{term}}}
\newcommand{\gammatype}{\Gamma^{\textrm{type}}}
\newcommand{\gammadata}{\Gamma^{\textrm{data}}}
\newcommand{\gammarhs}{\Gamma^{\textrm{rhs}}}
\newcommand{\gammanew}{\Gamma^{\prime}}
\newcommand{\gammadatarhs}[1]{\Gamma_{#1}}

\newcommand{\maxTerm}{n}
\newcommand{\maxType}{m}
\newcommand{\maxData}{o}
\newcommand{\maxArg}{k}
\newcommand{\maxConstr}{l}

\begin{figure}[!ht]
    \centering
    \begin{minipage}[t]{15cm}
    \centering
    \begin{displaymath}
    \begin{array}{lll}
  \multicolumn{3}{l}{}\\
  \multicolumn{3}{l}{d = \datatype{X}{(\seq{Y :: K})}{x}{(\seq{c})}} \\
  \multicolumn{3}{l}{c = x(\seq{T})}\\
  \\
  \multicolumn{3}{l}{\textsc{Χρήσιμες συναρτήσεις}}\\
  \branchTy{c}{R}
  &=& \seq{T} \rightarrow R \\
  \scottTy{d}
  &=& \lambda (\seq{Y::K}) . \forall R . (\seq{\branchTy{c}{R}}) \rightarrow R  \\
  \dataKind{d}
  &=& \seq{K} \kindArrow \Type \\
  \constrTy{d}{c}
  &=& \forall (\seq{Y::K}). \seq{T} \rightarrow X\ \seq{Y}\\
  \matchTy{d}
  &=& \forall (\seq{Y::K}) . (X\ \seq{Y}) \rightarrow (\scottTy{d}\ \seq{Y})\\
  \\
  \multicolumn{3}{l}{\textsc{Binder functions}}\\
  \dataBind{d}
  &=& X :: \dataKind{d}\\
  \constrBind{d}{c}
  &=& c : \constrTy{c}{X\ \seq{Y}}\\
  \constrBinds{d}
  &=& \seq{\constrBind{d}{c}}\\
  \matchBind{d}
  &=& x : \matchTy{d}\\
  \binds{x : T = t}&=&x:T\\
  \binds{X : K = T}&=&X:K\\
  \binds{d}&=& \dataBind{d}, \constrBinds{d}, \matchBind{d}\\
  &=& x : \matchTy{d}\\
    \end{array}
    \end{displaymath}
    \end{minipage}
    \caption{Βοηθητικοί ορισμοί}
    \label{fig:fir_aux}
\end{figure}

\newcommand{\provesok}{\vdash_{\textsf{ok}}}
\begin{figure}[!ht]
    \centering
    \begin{minipage}[t]{15cm}
    \centering
    \begin{displaymath}
    \begin{array}{ll}
    \inference[W-Con]{c = x(\seq{T}) & \seq{\Gamma \vdash T::\Type}}{\Gamma \provesok c} \\
    \\
    \inference[W-Term]{
      \Gamma \vdash T :: \Type & 
      \Gamma \vdash t : T}{\Gamma \provesok x : T = t} &
    \inference[W-Type]{\Gamma \vdash T :: K}{\Gamma \provesok X : K = T}\\
    \\
    \multicolumn{2}{l}{\inference[W-Data]{
      d=\datatype{X}{(\seq{Y :: K})}{x}{(\seq{c})} \\
      \Gamma^\prime = \Gamma, \seq{Y::K} &
      \seq{\Gamma^\prime \provesok c}}{\Gamma \provesok d}}\\
    \end{array}
    \end{displaymath}
    \end{minipage}
    \caption{Ορθή κατασκευή των δηλώσεων \texttt{let} }
    \label{fig:fir_wellformed}
\end{figure}

\begin{figure}[!ht]
    \centering
    \begin{minipage}[t]{15cm}
    \centering
    \begin{displaymath}
    \begin{array}{ll}
    \inference[Q-Refl]{}{T \equiv T} &
    \inference[Q-Symm]{T \equiv S}{S \equiv T}  \\
    \\
    \inference[Q-Trans]{S \equiv U & U \equiv T}{S \equiv T} &
    \inference[Q-Arrow]{S_1 \equiv S_2 & T_1 \equiv T_2}{(S_1 \typeArrow T_1) \equiv (S_2 \typeArrow T_2)} \\
    \\
    \inference[Q-All]{S \equiv T}{(\forall X::K.S) \equiv (\forall X::K.T)} &
    \inference[Q-Abs]{S \equiv T}{(\lambda X::K.S) \equiv (\lambda X::K.T)} \\
    \\
    \inference[Q-App]{S_1 \equiv S_2 & T_1 \equiv T_2}{S_1 T_1 \equiv S_2 T_2} &
    \inference[Q-Beta]{}{(\lambda X::K.T_1)T_2 \equiv \subst{X}{T_2}{T_1}}
    \end{array}
    \end{displaymath}
    \end{minipage}
    \caption{Ισοδυναμία τύπων της \FIR}
    \label{fig:fir_typeq}
\end{figure}

\begin{figure}[!ht]
    \centering
    \begin{minipage}[t]{15cm}
    \centering
    \begin{displaymath}
    \begin{array}{ll}
    \inference[T-Var]{x:T \in \Gamma}{\Gamma \vdash x:T}  &
    \inference[T-Abs]{\Gamma, x:T_1 \vdash t:T_2 & \Gamma \vdash T_1 :: \Type}{\Gamma \vdash (\lambda x:T_1.t) : T_1 \typeArrow T_2} \\
    \\
    \inference[T-App]{\Gamma \vdash t_1 : T_1 \typeArrow T_2 & \Gamma \vdash t_2 : T_1}{\Gamma \vdash (t_1 ~ t_2) : T_2} &
    \inference[T-TAbs]{\Gamma, X::K \vdash t:T }{\Gamma \vdash (\Lambda X::K.t) : (\forall X::K.T)} \\
    \\
    \inference[T-TApp]{\Gamma \vdash t_1: \forall X::K_2.T_1  & \Gamma \vdash T_2 :: K_2} {\Gamma \vdash (t_1 ~\{T_2\}) : \subst{X}{T_2}{T_1}} &
    \inference[T-Eq]{\Gamma \vdash t : S & S \equiv T}{\Gamma \vdash t : T} \\
    \\
    \multicolumn{2}{l}{\fomfDiff{\inference[T-Wrap]{\Gamma \vdash M: (F ~( \lambda (X :: K). \ifix F ~X)) ~T & \Gamma \vdash T:: K \\ \Gamma \vdash F :: (K\kindArrow\Type)\kindArrow (K\kindArrow\Type)}
            {\Gamma \vdash \wrap ~ F ~ T ~ M : \ifix F ~T} } }\\
    \\
    \multicolumn{2}{l}{\fomfDiff{\inference[T-Unwrap]{\Gamma \vdash M : \ifix F ~T & \Gamma \vdash T :: K }
            {\Gamma \vdash \unwrap M : (F ~( \lambda (X :: K). \ifix F ~X)) ~T  } } }\\
    \\
    \multicolumn{2}{l}{\firDiff{\inference[T-Let]{
    \Gamma \provesok \seq{b} &
    \Gamma \vdash T :: \Type &
    \Gamma, \seq{\binds{b}} \vdash t : T
    }
    { \Gamma \vdash (\tlet \seq{b} \tin t) : T
    }}}\\
    \\
    \multicolumn{2}{l}{\firDiff{\inference[T-LetRec]{
    \Gamma, \seq{\binds{b}} \provesok \seq{b} &
    \Gamma \vdash T :: \Type &
    \Gamma, \seq{\binds{b}} \vdash t : T
    }
    { \Gamma \vdash (\tlet \rec \seq{b} \tin t) : T
    }}}
    \end{array}
    \end{displaymath}
    \end{minipage}
    \caption{Σύνθεση τύπων της \FIR}
    \label{fig:fir_typing}
\end{figure}







\subsection{Kinding της \FOMF{}}
Η μόνη προσθήκη της \FOMF{} και \FIR{} στα kinds της \FOM{} είναι ο κανόνας
για το $\ifix$. Ο τελεστής σταθερού σημείου είναι πλήρως εφαρμοσμένος, δέχεται
τον pattern functor και τις παραμέτρους  του, ``δένει την θηλιά" αναδρομικά και επιστρέφει
έναν όρο με αυτόν τον τύπο, για αυτό και το αποτέλεσμα έχει kind $\Type$. Οι υπόλοιποι
κανόνες είναι αναμενόμενοι και αντίστοιχοι του λ-λογισμού με απλούς τύπους, υποστηρίζοντας
εφαρμογή, δημιουργία συνάρτησης και καθολικούς τύπους.

\vspace{1cm}
