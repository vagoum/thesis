Στο κεφάλαιο αυτό θα ορίσουμε την γραμματική και τους κανόνες σύνθεσης και ισοδυναμίας τύπων
της γλώσσας την γλώσσα \FIR{} ως μια προέκταση του συστήματος \FOMF. Ο λογισμός \FOMF
~αποτελείται από το \FOM, τον συνδυασμό των αξόνων του λάμδα κύβου (lambda cube, \cite{lambdacube})
που αντιστοιχούν στον πολυμορφισμό και στις συναρτήσεις τύπων. Τα τελευταία αποτελούν αρκετά
χρήσιμα στοιχεία σε μια γλώσσα που υποστηρίζει χαρακτηριστικά υψηλότερης τάξης. Συγκεκριμένα,
η ενδιάμεση αναπαράσταση που χρησιμοποιεί ο μεταγλωττιστής GHC της γλώσσας Haskell, εν ονόματι GHC Core,
είναι συνδυασμός της \FOM μαζί με πρόσθετους κανόνες για την αποδοτική κωδικοποίηση αλγεβρικών
τύπων δεδομένων.

Στην \FOMF ~οι αλγεβρικοί τύποι δεδομένων που συναντάμε στις δημοφιλείς γλώσσες προγραμματισμού
μπορούν να κωδικοποιηθούν εμμέσως, χωρίς να χρειάζεται να προσθέσουμε επιπλέον χαρακτηριστικά.
Ο τρόπος κωδικοποίησης των τύπων δεδομένων όμως είναι αρκετά δυσνόητος, γεγονός που δυσχεραίνει
τον χρήστη της γλώσσας. Αυτό δεν αποτελεί πρόβλημα για την \FOMF , που προορίζεται για ένα σύστημα ``χαμηλού" επιπέδου, το οποίο δεν θα διαβάζεται ή γράφεται από τον χρήστη, αλλά θα παράγεται αυτόματα
κατά την μεταγλώττιση από τον κώδικα Haskell.

Όπως αναφέρθηκε και στην παράγραφο \ref{subsec:plutus}, ο κώδικας Haskell που γράφει ο
προγραμματιστής των συμβολαίων μεταγλωττίζεται από ένα GHC plugin σε \FIR{} και στη συνέχεια
σε Plutus Core. Η γλώσσα Plutus Core είναι άμεση επέκταση της \FOMF. Η χρήση μιας απλής γλώσσας
σαν typed bytecode κάνει πιο εύκολη την χρήση τυπικών μεθόδων ανάλυσης και επαλήθευσης.

Καθώς το βήμα της μεταγλώττισης από την Haskell στην \FOMF{} είναι μεγάλο, είναι εύκολο να υπάρξει
λάθος κατά την μεταγλώττιση. Επίσης η \FOMF{} κάνει κάποιες βελτιστοποιήσεις, όπως την
εξάλειψη των αδρανών \texttt{let} αρκετά πιο δύσκολη, σε σύγκριση με μία γλώσσα που υποστηρίζει
let-bindings. Στο κεφάλαιο \ref{chap:chapter6} γίνεται αναφορά σε δύο τέτοιες βελτιστοποιήσεις.

Στο κεφάλαιο \ref{sec:fomf} θα παρουσιαστούν οι κανόνες σύνταξης και τύπων της \FOMF. Στη συνέχεια,
στο κεφάλαιο  \ref{sec:fir}  προσθέτουμε αναδρομικά let-bindings στην \FOMF{} που υποστηρίζουν την
δήλωση αμοιβαία αναδρομικών όρων και τύπων.
