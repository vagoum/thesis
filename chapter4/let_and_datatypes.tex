\section{Όροι Let} \label{sec:fir}

Στο σχήμα \ref{fig:fir_aux} υπάρχουν οι βοηθητικές συναρτήσεις που θα
χρησιμοποιήσουμε κατά την μελέτη των datatypes και για την μεταγλώττιση τους
από \FIR{} σε \FOMF{}, στο επόμενο κεφάλαιο.

Το επιπλέον στοιχείο που προσθέτει η \FIR{} είναι οι όροι $\tlet$ με τους
οποίους μπορούμε να ορίσουμε bindings όρων και τύπων δεδομένων, για χρήση στο
σώμα του $\tlet$. Οι κανόνες που υπαγορεύουν την σωστή κατασκευή των όρων
$\tlet$ φαίνονται στο σχήμα \ref{fig:fir_wellformed}.  Οι κανόνες τύπου που
δίνουμε για τους όρους αυτούς (σχήμα \ref{fig:fir_typeq}, κανόνες T-Let και
T-LetRec)

Διαισθητικά, υποθέτουμε ότι οι datatypes έχουν τους τύπους που δόθηκαν, και
πραγματοποιούμε τον έλεγχο τύπων με αυτούς.


\subsection{Datatypes}

Η \FIR{} υποστηρίζει δηλώσεις \emph{τύπων δεδομένων}. Ένας ορισμός τύπου
δεδομένων στην \FIR{} περιλαμβάνει αρχικά μία δήλωση τύπου, μαζί με τις
παραμέτρους τους, όπου οι δηλώσεις να συνοδεύονται με τα kinds τους. Στο δεξί
μέλος έχουμε τους \emph{κατασκευαστές} (constructors) και το όνομα της
συνάρτησης που καταστρέφει τον τύπο δεδομένων, \emph{συνάρτηση ταιριάσματος}. Η
σύνταξη των δηλώσεων τύπων δεδομένων μοιάζει επομένως αρκετά με το πως μπορεί ο
προγραμματιστής να δηλώσει τύπους δεδομένων στην Haskell.

Η συνάρτηση ταιριάσματος είναι ο τρόπος που χρησιμοποιούμε τους τύπους
δεδομένων στην γλώσσα μας, και πραγματοποιεί το ταίριασμα προτύπου (pattern
matching). Άλλη εναλλακτική θα ήταν να εισάγουμε έναν \texttt{typecase}
τελεστή, όπως έχει γίνει σε άλλες υλοποιήσεις της \FOMF{} που ξέρουμε από την
βιβλιογραφία \cite{cai}.

Παρακάτω βλέπουμε την δήλωση του κλασσικού τύπου δεδομένων $\Maybe$, που
``τυλίγει'' μια τιμή οποιουδήποτε τύπου, που μπορεί να είναι κενή. Οι
κατασκευαστές του είναι οι $\Nothing$, χωρίς ορίσματα, και  $\Just$ με ένα
όρισμα, και το όνομα της συνάρτησης ταιριάσματος είναι $\texttt{matchMaybe}$.
\begin{displaymath} \datatype{\Maybe}{(A ::
\Type)}{\textsf{matchMaybe}}{(\Nothing (), \Just (A))} \end{displaymath}

Ο τύπος της συνάρτησης $\texttt{matchMaybe}$ είναι $\Maybe A \rightarrow
\forall R . R \rightarrow (A \rightarrow R) \rightarrow R$ και υλοποιεί το
ταίριασμα σε τιμές του τύπου $\Maybe$ , όπως είδαμε στην παράγραφο
\ref{sec:data-encoding} όπου γίνεται λόγος για την κωδικοποίηση Scott. Η
συνάρτηση ταιριάσματος μετατρέπει τον αφηρημένο τύπο δεδομένων (abstract data
type) στον κωδικοποιημένο κατά Scott τύπο που μπορεί να κάνει χρήση των τιμών
του. Η ακριβής μορφή της συνάρτησης φαίνεται στην παράγραφο
\ref{sec:non-recursive-data} και ο τύπος της δίνεται από την βοηθητική
συνάρτηση $\matchTy{\Maybe}$ του σχήματος \ref{fig:compile-datatypes}

Τέλος, η υποστήριξη αναδρομικών δηλώσεων τύπων στην \FIR{},  κάνει τα $\ifix$,
$\wrap$ και $\unwrap$ περιττά από πλευρά εκφραστικότητας. Για πρακτικούς λόγους
όμως είναι προτιμότερο να παραμείνουν στην γλώσσα, παρά την επικάλυψη τους,
καθώς έτσι η \FIR{} αποτελεί γνήσιο υπερσύνολο της \FOMF{} που απλοποιεί την
παρουσίαση της.

\begin{figure}[!ht]
    \centering
    \begin{minipage}[t]{15cm}
    \centering
    \begin{displaymath}
    \begin{array}{ll}
    \inference[K-TVar]{X::K \in \Gamma}{\Gamma \vdash X :: K} &
    \inference[K-Abs]{\Gamma, X::K_1 \vdash T :: K_2}{\Gamma \vdash (\lambda X::K_1.T) :: K_1 \kindArrow K_2} \\
    \\
    \inference[K-App]{\Gamma \vdash T_1 :: K_1 \kindArrow K_2 \\ \Gamma \vdash T_2 :: K_1}{\Gamma \vdash (T_1 ~ T_2) :: K_2} &
    \inference[K-Arrow]{\Gamma \vdash T_1 :: \Type & \Gamma \vdash T_2 :: \Type}{\Gamma \vdash (T_1 \rightarrow T_2) :: \Type} \\
    \\
    \inference[K-All]{\Gamma, X::K \vdash T :: \Type}{\Gamma \vdash (\forall X::K.T) :: \Type} &
    \fomfDiff{\inference[K-Ifix]{\Gamma \vdash T:: K \\ \Gamma \vdash F :: (K \kindArrow \Type) \kindArrow (K \kindArrow \Type)} {\Gamma \vdash (\ifix ~ F ~ T) :: \Type} } 
    \end{array}
    \end{displaymath}
    \end{minipage}
    \caption{Kinding της \FIR}
    \label{fig:fir_kinding}
\end{figure}



\subsection{Ορθή κατασκευή των bindings και κατασκευαστών}

Ο έλεγχος τύπων απαιτεί οι όροι let να έχουν την σωστή μορφή, η οποία
περιγράφεται στο σχήμα \ref{fig:fir_wellformed}. Ουσιαστικά αυτό σημαίνει ότι
θέλουμε οι κατασκευαστές των τύπων να παίρνουν ορίσματα μεταβλητές με ground
types, δηλαδή ορίσματα με τύπο $\Type$ (κανόνας W-Con).  Στη συνέχεια ο κανόνας
W-Data ελέγχει ότι μία δήλωση ενός datatype είναι ορθή, όταν οι κατασκευαστές
έχουν τον τύπο με τον οποίο έχουν δηλωθεί, στο περιβάλλον που περιέχει τα
ονόματα και τα ορίσματα του τύπου δεδομένων που ορίζεται.


\subsection{Αμοιβαία αναδρομικά let στο επίπεδο των όρων}

Η \FIR{} υποστηρίζει και την δήλωση αμοιβαία αναδρομικών όρων. Σε αυτή την
διπλωματική δεν θα ασχοληθούμε με την μεταγλώττιση (αμοιβαία) αναδρομικών όρων,
αλλά αποτελεί σημαντικό χαρακτηριστικό μίας γλώσσας, όχι μόνο σε πρακτικό
επίπεδο, αλλά και σε θεωρητικό, καθώς η υποστήριξη αμοιβαία αναδρομικών όρων σε
μοντέλο πρόθυμης αποτίμησης δεν χει εξερευνηθεί από την βιβλιογραφία. Για τον
χειρισμό των αναδρομικών όρων Η αντιμετώπιση των αναδρομικών όρων αποτελεί
σημαντικό μέρος της ευρύτερης εργασίας, της οποίας μέρος είναι η παρούσα
διπλωματική.
