\section{Συνεισφορά} Στην παραπάνω εργασία ορίσαμε μια μικρή αλλά και
ταυτόχρονα εκφραστική γλώσσα την οποία μεταγλωττίσαμε με σταδιακά περάσματα σε
γνωστά και μελετημένα συστήματα του λ-λογισμού.  Η γλώσσα αυτή, \FIR{}
προσθέτει let-bindings αμοιβαία αναδρομικές δηλώσεις όρων και τύπων δεδομένων
στην \FOMF, που αποτελεί το σύστημα του λ-λογισμού που χρησιμοποιούμε. Οι
αναδρομικοί τύποι υποστηρίζονται μέσω μιας παραλλαγής του fixpoint τελεστή και
οι τύποι δεδομένων εκφράζονται μέσω της κωδικοποίησης Scott.

Παρόλο που τα επιμέρους συστατικά έχουν αναφερθεί στο παρελθόν στην
βιβλιογραφία, δεν έχουν συνδυαστεί κατά αυτό τον τρόπο, και αποτελούν μια
άσκηση στην σχεδίαση μιας συναρτησιακής γλώσσας που μπορεί να λειτουργήσει σαν
ενδιάμεση αναπαράσταση, με κύριο χαρακτηριστικό της την μεταγλώττιση
αναδρομικών χαρακτηριστικών από υψηλού επιπέδου δηλώσεις, σε εφαρμογή του
fixpoint τελεστή, ομοιόμορφα τόσο στο επίπεδο των όρων, όσο και στο επίπεδο των
τύπων.

Η παρούσα διπλωματική αποτελεί κομμάτι της εργασίας που έγινε δεκτή στο
συνέδριο επιστήμης υπολογιστών με επίκεντρο την μαθηματική κατασκευή
προγραμμάτων, Mathematics of Program Construction ( MPC 2019).  Η εργασία αυτή
περιλαμβάνει: \begin{itemize} \item μεταγλώττιση αμοιβαία αναδρομικών τύπων σε
      τύπους της \FOMF{}.  \item μεταγλώττιση αμοιβαία αναδρομικών όρων σε
      όρους της \FOMF{}.  \item υλοποίηση του μεταγλωττιστή από \FIR{} σε
        \FOMF{} στην γλώσσα Agda, που υποστηρίζει εξαρτημένους τύπους.
\end{itemize} Η υλοποίηση του μεταγλωττιστή σε Agda είναι εγγενώς συνοδευόμενη
από τύπους, και οι όροι συνοδεύονται με την προέλευση των τύπων τους
(\cite{altenkirch}). Το γεγονός αυτό αποδεικνύει ότι η μεταγλώττιση διατηρεί τα
\emph{kind} και τους τύπους.


\subsection{Μειονεκτήματα}

Η κωδικοποίηση που χρησιμοποιήθηκε προσθέτει αισθητό φόρτο, εφαρμογή και χρήση
συναρτήσεων, ειδικά σε σύγκριση με μία υλοποίηση αναδρομικών τύπων όπου τα
δεδομένων συνοδεύονται με επιπλέον πεδία που δείχνουν στους αναδρομικούς
ορισμούς που χρησιμοποιούν. Στις γλώσσες που εξετάσαμε, οι τύποι δεδομένων
είναι καθαρές συναρτήσεις στο επίπεδο των τύπων, η γλώσσα δεν περιέχει πρόσθετα
χαρακτηριστικά για την υποστήριξη τους.

Σε μία γλώσσα που στοχεύει σε ανταγωνιστική απόδοση, συγκρίσιμη με αυτή των
mainstream γλωσσών προγραμματισμού, ο επιπλέον φόρτος που προστίθεται με την
κωδικοποίηση αυτή είναι απαγορευτικός.  Στη περίπτωση όμως που η ορθότητα μας
ενδιαφέρει περισσότερο, είναι ωφέλιμο να έχουμε μια μικρή, αυτοτελή και πολύ
εκφραστική γλώσσα-πυρήνα (core language) που είναι ενδελεχώς μελετημένη.

\subsection{Βελτιστοποιήσεις}

Έχουμε αναφέρει ότι η δυνατότητα υποστήριξης $\tlet$-δεσιμάτων ``ξεκλειδώνει''
και απλοποιεί βελτιστοποιήσεις του μεταγλωττιστή. Θα αναφέρουμε δύο τέτοιες
βελτιστοποιήσεις.


\subsection{Εξάλειψη αδρανών let-binding}

Σε μία γλώσσα με όρους $\tlet$ είναι αρκετά εύκολη η εφαρμογή της
βελτιστοποίησης που απαλείφει τα αδρανή let. Συγκεκριμένα το πέρασμα της
βελτιστοποίησης εντοπίζει τους όρους/τύπους/datatypes που έχουν δεσμευτεί στην
κορυφή του $\tlet$, αλλά δεν χρησιμοποιείται στο σώμα του. Τέτοια bindings
μπορούν να απαλειφθούν.

Αντιθέτως στην \FOMF{}, οι όροι $\tlet$ μεταφράζονται σε δημιουργία και
εφαρμογή συναρτήσεων, περιπλέκοντας την απλή ``προστακτική'' φύση των όρων που
κάνουν την εξάλειψη των αδρανών δεσιμάτων τόσο φυσική στην \FIR{}.  Η εφαρμογή
της ίδιας βελτιστοποίησης στην \FOMF{} θα απαιτούσε πιο λεπτομερή μελέτη και
υλοποίηση της δομής του προγράμματος.

Ειδικά η κωδικοποίηση των αμοιβαία αναδρομικών τύπων δημιουργεί αρκετούς
επιπρόσθετους όρους στην \FOMF{}, καθιστώντας αρκετά δύσκολη ως αδύνατη την
εξάλειψή τους, σε αντίθεση με την περίπτωση της \FIR{} που είναι αρκετά απλή.


\subsection{Βελτιστοποίησης γνωστού κατασκευστή}

Η βελτιστοποίηση της \emph{περίπτωσης γνωστού κατασκευαστή}
(case-of-known-constructor optimization) είναι σημαντική για τις συναρτησιακές
γλώσσες (\cite{jones1998transformation}).  Συχνά όταν πραγματοποιούμε ένα
ταίριασμα προτύπου ξέρουμε την δομή της τιμής στην οποία θα πραγματοποιηθεί το
ταίριασμα, και μπορούμε να απαλείψουμε τις περιττές ενδιάμεσες κατασκευές και
να δουλέψουμε κατευθείαν στην τιμή. Ένα παράδειγμα της βελτιστοποίησης στην
γλώσσα \FIR{} είναι:
%
\begin{align*} \Match\ \{\Int\}\ (\Just\ \{\Int\}\ 1)\ 0\ (\lambda x . x+1)
\Longrightarrow (\lambda x . x+1)\ 1 \end{align*}
%
  Η βελτιστοποίηση αυτή είναι εύκολο να υλοποιηθεί στην \FIR{}, όπου έχουμε
  γνώση των συναρτήσεων ταιριάσματος και των constructors του κάθε τύπου
  δεδομένων.  Η βελτιστοποίηση αυτή δεν είναι εφαρμόσιμη όταν οι όροι $\tlet$
  μεταγλωττιστούν σε \FOMF{}.


    \section{Μελλοντικές Κατευθύνσεις}

   Περισσότερη δουλειά μπορεί να γίνει στην σημασιολογία της \FIR{}. Στα
   προηγούμενα κεφάλαια δεν ορίστηκε ξεχωριστά η σημασιολογία της \FIR{},
   αντιθέτως η σημασιολογία της υπαγορεύεται από την μετάφραση σε \FOMF{}.
   Δηλαδή η ``σημασία'' ενός προγράμματος \FIR{} ορίζεται ως η ``σημασία'' του
   προγράμματος \FOMF{} στο οποίο μεταφράζεται.

   Ένας άλλος άξονας στον οποίον θα μπορούσε να βελτιωθεί η εργασία είναι στην
   ορθότητα της μεταγλώττισης . Η μεταγλώττιση από την \FIR{} στην \FOMF{}
   είναι ορθή αν οι λειτουργίες της αποτίμησης και της μεταγλώττισης
   αντιμετατίθενται.  Δηλαδή η αποτίμηση ενός προγράμματος \FIR{} και η
   μεταγλώττιση σε \FOMF{}, ακολουθούμενη από την αποτίμηση πρέπει να δίνουν το
   ίδιο αποτέλεσμα για όλα τα προγράμματα.

   Επιπρόσθετα, μία πραγματική γλώσσα προγραμματισμού με χρήση στο blockchain
   εμπεριέχει και θεωρητικές ατέλειες, όπως υλοποίηση κρυπτογραφικών
   συναρτήσεων, και built-in τύπους. Η θεωρητική ανάλυση χρειάζεται να
   συμπεριλάβει και αυτές της θεωρητικά βαρετές, αλλά πρακτικά σημαντικές
   πτυχές της γλώσσας. Συγκεκριμένα η Plutus Core, η πραγματική εκδοχή της
   \FOMF{} υποστηρίζει sized integers, που κουβαλάνε μαζί το μέγεθός τους. Αυτό
   το χαρακτηριστικό στοχεύει στην ευκολότερη ανίχνευση λαθών που προκύπτουν
   από αριθμητικά overflows. Τα στοιχεία αυτά αποτελούν παραδείγματα των
   εμποδίων που πρέπει να λυθούν από μία γλώσσα που στοχεύει στην χρήση της σε
   πραγματικά συστήματα στο blockchain.
