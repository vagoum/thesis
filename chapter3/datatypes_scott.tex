Η κωδικοποίηση Scott ταυτίζει τον τύπο ενός datatype ως τον τύπο της συνάρτησης που κάνει
pattern match σε αυτόν. Για παράδειγμα για τον τύπο των Booleans έχουμε:
\begin{displaymath}
  \forall R . R \rightarrow R \rightarrow R
\end{displaymath}
Μια συνάρτηση που κάνει ταίριασμα (pattern matching) σε μία συνάρτηση τύπου
$\Bool$, πρέπει για κάθε τύπο αποτελέσματος $R$,
να μπορεί να δώσει ένα $R$ στην περίπτωση που η τιμή είναι \texttt{True}
και ένα $R$ στην περίπτωση του \texttt{False}. Στην γενική περίπτωση όπου οι
κατασκευαστές του τύπου
δεδομένων δέχονται παραμέτρους, πρέπει να ο τύπος να μπορεί να επιστρέψει $R$,
με είσοδο τα ορίσματα του κατασκευαστή.

Ο τύπος των φυσικών, $\NNat$, γίνεται:
\begin{displaymath}
  \forall R . R \rightarrow (\NNat \rightarrow R) \rightarrow R
  \end{displaymath}

Παρατηρούμε την εμφάνιση του $\NNat$ στον ορισμό, που χρησιμοποιείται από
τον αναδρομικό κατασκευαστή στην περίπτωση του \texttt{Suc}. Για να αποκτήσει
νόημα ο τελικός τύπος θα πρέπει να παντρευτεί με την αναδρομή στο επίπεδο
των τύπων που υποστηρίζει η γλώσσα μας, με τη χρήση του fixpoint τελεστή.

Η κωδικοποίηση Church του τύπου $\Bool$ , και κάθε μη αναδρομικού τύπου
ταυτίζεται με την Scott, αλλά είναι διαφέρουν στην περίπτωση αναδρομικών τύπων.
Η Church κωδικοποίηση των $\NNat$ είναι:
\begin{displaymath}
  \forall R . R \rightarrow (R \rightarrow R) \rightarrow R
  \end{displaymath}

Εδώ η αναδρομική εμφάνιση του $\NNat$ έχει απαλειφθεί και αντικατασταθεί
με $R$. Σε αντίθεση με την κωδικοποίηση Scott που αντιστοιχεί στο ταίριασμα
προτύπου πάνω στον τύπο, η κωδικοποίηση Church δίνει πρόσβαση στον πλήρη
αναδρομικό τύπο.

Συνοπτικά οι διαφορές των δύο κωδικοποιήσεων είναι οι εξής:
\begin{itemize}
\item Για να επεξεργαστούμε μια τιμή κωδικοποιημένη κατά Church, πρέπει
να ``σκανάρουμε'' (fold), ολόκληρη την δομή, που οδηγεί σε θέματα απόδοσης.
Για μια τιμή κωδικοποιημένη κατά Scott, αρκεί να κοιτάξουμε το τελευταίο
επίπεδο. Το θέμα αυτό ονομάζεται successor problem (από την δομή των φυσικών
αριθών στην Church κωδικοποίηση) και αναλύεται στο  \cite{scott}.

\item  Στην κωδικοποίηση Church η αναδρομική εμφάνιση είναι ήδη ``τυλιγμένη''
στην αναδρομή, επομένως δεν χρειαζόμαστε επιπλέον εργαλεία από την θεωρία
αναδρομικών τύπων για να την χρησιμοποιήσουμε. Αντιθέτως στην κωδικοποίηση
Scott, χρειαζόμαστε κατάλληλη αναδρομή στο επίπεδο των τύπων ώστε να
ερμηνεύσουμε κατάλληλα τον Scott τύπο.

\end{itemize}
