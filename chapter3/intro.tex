Σε αυτό το κεφάλαιο θα παρουσιάσουμε την προαπαιτούμενη θεωρία αναδρομικών
τύπων που χρειάζεται για να εκφράσουμε την \FOMF. Ακολουθώντας το \cite{tapl},
στις επόμενες παραγράφους θα δούμε τις διαφορετικές επιλογές που παρουσιάζονται
στον κατασκευαστή μιας γλώσσας με ισχυρό σύστημα τύπων με αναδρομή.

Στην συνέχεια θα συζητηθεί η επιλογή του τελεστή σταθερού σημείου που θα
χρησιμοποιηθεί για την υποστήριξη της αναδρομής στην γλώσσα και η κωδικοποίηση
των τύπων δεδομένων που ακολουθούμε. Στο τέλος του κεφαλαίου βρίσκεται η
σχετική βιβλιογραφία γύρω από τα θέματα που θα αναπτυχθούν.
