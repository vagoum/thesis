
Όπως αναφέρθηκε στην προηγούμενη παράγραφο, διαφορετικές προσεγγίσεις για την κωδικοποίηση
τύπων δεδομένων έχουν συζητηθεί στο \cite{scott}, μαζί με μία τυπική περιγραφή της κωδικοποίησης
Scott. Στην εργασία αυτή η κωδικοποίηση παρουσιάζεται πιο αναλυτικά, μαζί με πλήρες χειρισμό
των αναδρομικών τύπων.

Στο \cite{fixmutualgeneric} γίνεται λόγος για την χρήση \textit{σταθερών σημείων με δείκτη} (indexed 
fixpoints), στην ανάπτυξη τεχνικών generic programming. Η παραπάνω δουλειά επεκτείνεται στο 
\cite{genericwithindexed} ώστε να υποστηρίζουν παραμετροποηιμένους τύπους.

Μια άλλη υλοποίηση της \FOMF~ με ισοαναδρομικούς τύπους παρουσιάζεται στο \cite{BrownP17}.
Περιλαμβάνει και έναν  τελεστή που πραγματοποιεί το pattern matching στους τύπους, εξυπηρετώντας
αντίστοιχο ρόλο με την match function στον ορισμό ενός τύπου δεδομένων στην \FIR{}, όπως θα
δούμε στη συνέχεια. Χρησιμοποιούν επίσης παρόμοιο τελεστή σταθερού σημείου, μόνο για
την περίπτωση που ο δείκτης έχει kind $\Type$, ενώ ο τελεστής $\ifix$ που παρουσιάσαμε
παραπάνω λειτουργεί για κάθε kind $k$. 