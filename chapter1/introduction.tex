\section{Σκοπός}

Η εργασία αυτή αποσκοπεί στον ορισμό μιας συναρτησιακής ενδιάμεσης γλώσσας προγραμματισμού για
χρήση από μεταγλωττιστές, για χρήση ως ενδιάμεσο βήμα κατά την μεταγλώττιση από μία γλώσσα προέλευσης
σε μία γλώσσα στόχο. Η γλώσσα αυτή αποτελεί προέκταση του απλού λ-λογισμού, εμπλουτισμένο με πολυμορφισμό, αναδρομικούς τύπους μεγαλύτερης τάξης και συναρτήσεις στο επίπεδο των τύπων και των
kinds. Επιπλέον, η γλώσσα υποστηρίζει αναδρομικά $\tlet$ στο επίπεδο των όρων και στο επίπεδο των τύπων, χωρίς να βασίζεται στην οκνηρή αποτίμηση του λ-λογισμού στη βάση της,
  προσέγγιση που δεν έχει εξερευνηθεί επιτυχώς στην βιβλιογραφία μέχρι σήμερα.

  Στην συνέχεια δείχνουμε πώς μπορούν να μεταγλωττιστούν χαρακτηριστικά της παραπάνω ενδιάμεσης
  γλώσσας στο σύστημα \FOMF ~(\FOM ~εμπλουτισμένο με αναδρομικούς τύπους). Συγκεκριμένα στην
  παρούσα εργασία επικεντρωνόμαστε στην μεταγλώττιση των αναδρομικών και αμοιβαία αναδρομικών τύπων
  δεδομένων. Τέλος θα δούμε την υλοποίηση αυτής της μεταγλώττισης όπως είναι υλοποιημένη σε πραγματικό
  μεταγλωττιστή σε Haskell.

  \section{Κίνητρο}

  Οι περισσότεροι μεταγλωττιστές κάνουν χρήση ενδιάμεσων αναπαραστάσεων κατά την μεταγλώττιση του
  πηγαίου κώδικα από την προερχόμενη γλώσσα στην γλώσσα στόχο. Αποτελεί συνήθη πρακτική, ειδικά για
  τους μεταγλωττιστές συναρτησιακών γλωσσών, η ενδιάμεση αυτή γλώσσα να είναι επέκταση κάποιας
  εκδοχής του λ-λογισμού, λόγω της απλότητας και εκφραστικότητας της αναπαράστασης.

  Ταυτόχρονα όμως, η χρήση τέτοιων γλωσσών έχει το μειονέκτημα ότι καθώς προορίζεται για χρήση από το
  επόμενο στάδιο της μεταγλώττισης, συνήθως για βελτιστοποίηση και επιπλέον περάσματα, δεν είναι εύκολο
  να κατανοηθεί και να γραφεί από άνθρωπο. Επίσης το βήμα της μεταγλώττισης γίνεται συνήθως αρκετά
  μεγάλο, γεγονός που καθιστά συμφέρουσα την ανάλυση της μεταγλώττισης σε μικρότερα βήματα.

  Η χρήση μιας ενδιάμεσης γλώσσας με ισχυρούς τύπους έχει προταθεί παλαιότερα (\cite{henk}). Οι εν λόγω
  συγγραφείς βασίζουν την γλώσσα τους σε λ-λογισμό με εξαρτημένους τύπους. Για την ανάπτυξη των τεχνικών που θα παρουσιαστούν δεν είναι αναγκαία η επιστράτευση τόσο δυνατού λογισμού. Κατά αυτόν τον τρόπο λοιπόν κάνουμε τις λιγότερες δυνατές προσθήκες που επιτρέπουν την έκφραση των επιθυμητών 
  χαρακτηριστικών στην γλώσσα.

  Η ενδιάμεση γλώσσα που ορίζουμε επιτρέπει την χρήση $\tlet$-bindings για όρους και για τύπους δεδομένων,
  προσφέροντας δομές προγραμματισμού πιο κοντά στον χρήστη.

  Η γλώσσα που παρουσιάζουμε χρησιμοποιείται σαν ενδιάμεση αναπαράσταση στην αρχιτεκτονική Plutus,
  μια πλατφόρμα έξυπνων συμβολαίων βασισμένο στο Cardano blockchain. Σύντομη περιγραφή της
  αρχιτεκτονικής γίνεται στο κεφάλαιο \ref{subsec:plutus}.

  \section{Δομή της εργασίας}

  Αρχικά γίνεται μια περιγραφή του τομέα εφαρμογής των κύριων ιδεών της εργασίας,
  δηλαδή του blockchain και των έξυπνων συμβολαίων (\ref{chap:chapter2}.

     Στο κεφάλαιο \ref{chap:chapter2}  παρουσιάζεται η απαραίτητη θεωρία γλωσσών και αναδρομικών τύπων
      για την παρουσίαση των τεχνικών των επόμενων κεφαλαίων.

      Στη συνέχεια έχουμε τον ορισμό και τις ιδιότητες της γλώσσας \FOMF{}, άμεσα προερχόμενη από
      τον λ-λογισμό και της \FIR{}, που επεκτείνει την \FOMF{} με $\tlet$-bindings (κεφάλαιο \ref{chap:chapter4})

        Προχωρώντας στο κεφάλαιο \ref{chap:chapter5} γίνεται η μεταγλώττιση των αμοιβαία αναδρομικών τύπων
        δεδομένων από την \FIR{} στην \FOMF{}. Ο χειρισμός των αμοιβαία αναδρομικών τύπων απαιτεί
        την χρήση τεχνικών από την βιβλιογραφία γύρω από τον \emph{γενικευμένο προγραμματισμό} (generic
          programming). Συγκεκριμένα, θα  συζητηθούν οι \emph{δεικτοδοτημένοι τελεστές σταθερού σημείου}
        (indexed fixpoints) και ετικέτες στο επίπεδο των τύπων (type-level tags) ώστε να συνδυαστούν
        διαφορετικοί αμοιβαία αναδρομικοί τύποι σε έναν κοινό τύπο. Η τεχνική αυτή αναφέρεται στο
        \cite{fixmutualgeneric}, παρόλα αυτά η προσέγγιση που ακολουθείται εδώ είναι διαφορετική,
        καθώς βάζει στην εξίσωση και την ``ορθότερη" κωδικοποίηση Scott, και έχουμε σαν στόχο την \FOMF{},
        σε αντίθεση με μία γλώσσα με πλήρης εξαρτημένους τύπους (dependent types), όπως οι παραπάνω
        συγγραφείς.

       Καταλήγοντας,  στο κεφάλαιο \ref{chap:chapter6} αναφέρονται  συμπεράσματα, μειονεκτήματα και μελλοντικές κατευθύνσεις αυτής της δουλειάς, μαζί με παραδείγματα κάποιων βελτιστοποιήσεων που ενεργοποιεί η
        χρήση $\tlet$ όρων στην γλώσσα.
